\chapter{Kombinatorická hra}
\label{sec:Games}

\section{Teoretický základ}
\label{sec:GamesTheory}

Kombinatorická hra dvou hráčů na orientovaném grafu je další příklad P-úplného problému \cite{sawa}. Hra má tyto vlastnosti:

\begin{itemize}
    \item \textbf{Dva hráči:} Každé pole grafu má specifikováno, který hráč je na tahu – Hráč I (první hráč) nebo Hráč II (druhý hráč).
    \item \textbf{Konečná pozice:} Hra končí, když je hráč na tahu v pozici bez možných dalších tahů.
\end{itemize}

Problém spočívá v rozhodnutí, zda Hráč I má výherní strategii ze zadané počáteční pozice:

\begin{itemize}
    \item \textbf{Vstup:} Orientovaný graf (DG), kde každý uzel reprezentuje herní pozici přiřazenou některému z hráčů, hrany reprezentují možné tahy a jeden uzel je označen jako počáteční pozice.
    \item \textbf{Výstup:} Rozhodnutí, zda Hráč I má výherní strategii ze startovní pozice.
\end{itemize}

\textbf{Pravidla hry:} Hra končí, když se dostane do pozice, kde hráč na tahu nemá žádný možný tah – tento hráč pak prohrává. Hráči se střídají v tazích podle toho, jaký hráč je přiřazen aktuální pozici: je-li uzel přiřazen Hráči I, táhne Hráč I; je-li přiřazen Hráči II, táhne Hráč II.

\textbf{Výherní strategie:} Výherní strategie pro Hráče I je taková posloupnost tahů, která garantuje výhru Hráče I bez ohledu na to, jak hraje Hráč II. Jinými slovy, Hráč I má výherní strategii, pokud existuje způsob, jak vždy volit tahy tak, aby se hra dostala do pozice, kde je Hráč II na tahu a nemá žádný možný tah \cite{sawa}.

Hráč I tedy vyhrává v těchto případech:
\begin{itemize}
    \item Může vynutit situaci, kdy se hra dostane do pozice, kde je Hráč II na tahu a nemá žádný možný tah (terminální pozice pro Hráče II).
    \item Hra začíná v pozici přiřazené Hráči II, která nemá žádné odchozí hrany – Hráč II okamžitě prohrává a Hráč I vyhrává.
\end{itemize}

\subsection{P-úplnost problému}
\label{sec:GamesCompleteness}

Tento problém je P-úplný \cite{sawa, miyano}. Určení výherní strategie lze provést v polynomiálním čase pomocí tzv. retrográdní analýzy \cite{sawa}, která zpětně vyhodnocuje pozice od koncových uzlů. Algoritmus pracuje iterativně a dokáže zpracovat i grafy s cykly – v případě cyklů bez jednoznačného výsledku přiřadí pozicím status remízy. 

Logika rozhodování funguje následovně: Pro každou pozici určujeme, zda je výherní pro hráče, který je v ní na tahu. Pokud má hráč v dané pozici alespoň jeden tah do pozice, která je výherní pro něj, pak je i aktuální pozice výherní pro něj. Naopak, pokud všechny možné tahy vedou do pozic výherních pro protihráče, pak je aktuální pozice výherní pro protihráče (tedy prohrávající pro hráče na tahu). V případě, že některé tahy vedou do pozic výherních pro protihráče a některé do pozic, jejichž výsledek ještě nebyl určen, zůstává aktuální pozice také nerozhodnuta (remíza).

Implementace využívá frontu (queue) pro postupné zpracování pozic – začínáme koncovými pozicemi a postupně šíříme výsledky směrem k počáteční pozici. Liší se od algoritmu vyhodnocování MCVP (viz kapitola \ref{sec:MCVPEvaluation}) tím, že místo kombinace logických hodnot zde pracujeme s výherními stavy jednotlivých hráčů. I když je problém řešitelný v třídě P, jeho P-úplnost naznačuje, že pravděpodobně neexistuje efektivní paralelní algoritmus – řešení vyžaduje sekvenční zpracování pozic.

\section{Formát vstupu}
\label{sec:GamesInput}

Aplikace nabízí tři způsoby zadávání herního grafu:

\begin{itemize}
    \item \textbf{Interaktivní editace:} Uživatel vytváří a upravuje graf pomocí grafického editoru (viz sekce \ref{sec:GamesInteractive}).
    \item \textbf{Generování náhodných her:} Automatické vytvoření náhodného grafu podle zadaných parametrů (viz sekce \ref{sec:GamesGeneration}).
    \item \textbf{Předpřipravené sady:} Načtení ukázkových připravených příkladů (viz sekce \ref{sec:GamesPreparedSets}).
\end{itemize}

Všechny metody využívají komponentu \texttt{DisplayGraph} pro vizualizaci herního grafu (viz sekce \ref{sec:GamesVisualization}).  

\section{Interaktivní editace grafu}
\label{sec:GamesInteractive}

Komponenta \texttt{ManualInput} umožňuje vytvářet a upravovat herní grafy pomocí interaktivního grafického editoru. Uživatel může:

\begin{itemize}
    \item \textbf{Přidávat uzly:} Vytvořit novou pozici.
    
    \item \textbf{Upravovat uzly:} Kliknutím na uzel ho označíme. Označený uzel můžeme:
    \begin{itemize}
        \item Změnit hráče který je na tahu (Hráč I nebo Hráč II)
        \item Odstranit pozici z grafu
        \item Nastavit jako počáteční pozici
        \item Použít jako zdroj nebo cíl pro vytvoření hrany
        \item Vytvořit hranu z nebo do tohoto uzlu
        \item Odstranit hrany k nebo od tohoto uzlu
    \end{itemize}
            
    \item \textbf{Reorganizovat graf:} Uzly lze přesouvat myší pro lepší vizuální uspořádání.
\end{itemize}

Aplikace průběžně validuje graf. Uživatel je upozorněn, pokud není nastavena počáteční pozice, což je nutné pro spuštění analýzy.

\section{Generování náhodných her}
\label{sec:GamesGeneration}

Modul \texttt{Generator.js} obsahuje funkci \texttt{generateGraph()} pro vytváření náhodných herních grafů. Uživatel nastavuje dva parametry:

\begin{itemize}
    \item \textbf{Počet pozic:} Kolik uzlů (herních pozic) bude graf obsahovat.
    \item \textbf{Pravděpodobnost hrany:} Hodnota 0\% -- 100\% určující, jak pravděpodobné je vytvoření hrany mezi dvěma uzly.
\end{itemize}

Algoritmus generování probíhá ve dvou fázích:

\begin{enumerate}
    \item \textbf{Vytvoření kostry:} Vytvoříme uzly očíslované 0 až \( n-1 \) a každému náhodně přiřadíme hráče. Pro každý uzel \( i > 0 \) pak vytvoříme hranu z náhodného předchozího uzlu (s indexem menším než \( i \)) do uzlu \( i \). Tím zajistíme, že uzel 0 (počáteční pozice) může dosáhnout všechny ostatní uzly, a současně vytvoříme acyklickou kostru grafu.
    
    \item \textbf{Přidání dalších hran:} Pro každou dvojici uzlů \( i, j \) (kde \( i \neq j \)), podle dané pravděpodobnosti přidáme hranu \( i \to j \), pokud ještě neexistuje. Hrany mohou být přidány v libovolném směru, což umožňuje vznik cyklů v grafu.
\end{enumerate}

Výsledný graf je vždy souvislý s uzlem 0 jako počáteční pozicí. Graf může obsahovat cykly, což odpovídá reálným herním situacím, kde se hra může dostat do opakujících se pozic. Algoritmus analýzy (viz sekce \ref{sec:GamesAlgorithm}) je navržen tak, aby správně zpracoval i grafy s cykly a přiřadil těmto pozicím status remízy, pokud nelze jednoznačně určit výherce.

\section{Předpřipravené sady}
\label{sec:GamesPreparedSets}

Ve složce \texttt{Sady/CombinatorialGame} najdeme předpřipravené herní grafy různé velikosti a složitosti.

\subsection{Ukládání a načítání her}
\label{sec:GamesSerialization}

Herní grafy lze exportovat a importovat ve formátu JSON pomocí komponenty \texttt{FileTransferControls}. Formát obsahuje:

\begin{itemize}
    \item Pole \texttt{nodes} s uzly – každý má ID a přiřazeného hráče
    \item Pole \texttt{edges} s hranami ve formátu source-target
    \item \texttt{startingPosition} určující ID počáteční pozice
\end{itemize}

Tento formát umožňuje sdílení her mezi uživateli a vytváření knihovny předpřipravených příkladů.

\section{Vizualizace grafu}
\label{sec:GamesVisualization}

Vizualizace herního grafu využívá komponentu \texttt{DisplayGraph}, která využívá knihovnu \emph{react-force-graph-2d}. Graf zobrazuje:

\begin{itemize}
    \item \textbf{Uzly:} Každý uzel reprezentuje herní pozici. Pod uzlem je zobrazen hráč, který je na tahu (I nebo II). Při najetí myší na libovolný uzel se v centrech všech uzlů zobrazí jejich ID pro snadnější orientaci.
    
    \item \textbf{Počáteční pozice:} Označena oranžovou barvou.
    
    \item \textbf{Hrany:} Možné tahy jsou zobrazeny jako směrované hrany mezi uzly. Hrany patřící do výherní strategie jsou zvýrazněny tlustší čarou a žlutou barvou.
\end{itemize}

Graf používá fyzikální simulaci pro automatické rozmístění uzlů. Uživatel může uzly přesouvat myší a graf přibližovat nebo oddalovat kolečkem myši.

\section{Algoritmus analýzy}
\label{sec:GamesAlgorithm}

Řešení problému kombinatorických her je implementováno v modulu \texttt{ComputeWinner.js}. Algoritmus využívá iterativní přístup (retrográdní analýzu) pro postupné označování pozic od koncových uzlů směrem k počáteční pozici.

\subsection{Vyhodnocení výherních pozic}
\label{sec:GamesRetrograde}

Algoritmus určuje, zda je daná pozice výherní pro hráče, který je v ní na tahu, nebo zda skončí remízou (v případě cyklů bez vynuceného výsledku). Vyhodnocení probíhá následovně:

\begin{enumerate}
    \item \textbf{Koncové pozice:} Pozice bez dalších tahů jsou prohrávající pro hráče, který je v nich na tahu.
    
    \item \textbf{Zpětné šíření:} Od koncových pozic se postupně propagují výsledky:
    \begin{itemize}
        \item Pokud existuje tah do prohrávající pozice soupeře, aktuální pozice je vyhrávající.
        \item Pokud tentýž hráč pokračuje v tahu (bez změny hráče), tah do vyhrávající pozice znamená, že i původní pozice je vyhrávající.
        \item Pokud všechny tahy vedou do vyhrávajících pozic soupeře, aktuální pozice je prohrávající.
    \end{itemize}
    
    \item \textbf{Neurčené pozice:} Pozice, jejichž status nelze jednoznačně určit (například kvůli cyklům), zůstávají označeny jako remíza.
\end{enumerate}

Časová složitost je \( O(V + E) \), kde \( V \) je počet uzlů a \( E \) počet hran.

\subsection{Optimální tahy}
\label{sec:GamesOptimalMoves}

Funkce \texttt{getOptimalMoves()} identifikuje hrany, které jsou součástí výherní strategie. Hrana z pozice \( u \) do pozice \( v \) je optimální, pokud obě pozice jsou výherní pro Hráče I. Tyto hrany jsou zvýrazněny ve vizualizaci a ukazují uživateli cestu k výhře.

\section{Krokové vyhodnocení}
\label{sec:GamesStepByStep}

Pro detailnější průchod grafem implementuje komponenta \texttt{StepByStepGame} krokovatelnou analýzu. Algoritmus zaznamenává každý krok aktualizace stavu grafu:

\begin{itemize}
    \item \textbf{Inicializace:} Všechny pozice jsou na začátku ve stavu REMÍZA.
    \item \textbf{Terminální stavy:} Identifikace pozic bez tahů (PROHRA).
    \item \textbf{Aktualizace:} Postupné šíření výherních a prohrávajících stavů grafem.
    \item \textbf{Vysvětlení:} U každého kroku je zobrazeno vysvětlení, proč došlo ke změně stavu (např. „Z pozice X lze táhnout do prohrávající pozice Y").
\end{itemize}

Uživatel může procházet kroky analýzy pomocí navigačních tlačítek. Aktuálně aktualizovaný uzel je v grafu zvýrazněn. Tato funkce pomáhá pochopit, jak algoritmus postupně řeší hru a jak se vypořádává s cykly.
