\chapter{Kombinatorická hra}
\label{sec:Games}

\section{Teoretický základ}
\label{sec:GamesTheory}

Kombinatorická hra dvou hráčů na orientovaném grafu je další příklad P-úplného problému \cite{sawa}. Hra má tyto vlastnosti:

\begin{itemize}
    \item \textbf{Dva hráči:} Každé pole grafu má specifikováno, který hráč je na tahu – Hráč I (první hráč) nebo Hráč II (druhý hráč).
    \item \textbf{Konečná pozice:} Hra končí, když je hráč na tahu v pozici bez možných dalších tahů.
\end{itemize}

Problém spočívá v rozhodnutí, zda Hráč I má výherní strategii ze zadané počáteční pozice:

\begin{itemize}
    \item \textbf{Vstup:} Orientovaný acyklický graf (DAG), kde každý uzel reprezentuje herní pozici přiřazenou některému z hráčů, hrany reprezentují možné tahy a jeden uzel je označen jako počáteční pozice.
    \item \textbf{Výstup:} Rozhodnutí, zda Hráč I má výherní strategii ze startovní pozice.
\end{itemize}

Hráč I vyhrává, pokud se dostane do pozice, kde Hráč II nemá žádný možný tah. Naopak, Hráč I prohrává, když se sám dostane do pozice bez možných tahů \cite{papadimitriou1993}.

\subsection{P-úplnost problému}
\label{sec:GamesCompleteness}

Tento problém je P-úplný \cite{arora2009}. Určení výherní strategie lze provést v polynomiálním čase pomocí tzv. retrográdní analýzy, která zpětně vyhodnocuje pozice od koncových uzlů. Implementace je podobná algoritmu vyhodnocování MCVP (viz kapitola \ref{sec:MCVPEvaluation}) – opět používáme průchod do hloubky s memoizací. Liší se pouze logika rozhodování: místo kombinace logických hodnot zde určujeme, která pozice je výherní na základě možností volby jednotlivých hráčů. I když je problém řešitelný v třídě P, jeho P-úplnost naznačuje, že pravděpodobně neexistuje efektivní paralelní algoritmus – řešení vyžaduje sekvenční zpracování pozic.

%clean 

\section{Formát vstupu}
\label{sec:GamesInput}

Aplikace nabízí tři způsoby zadávání herního grafu:

\begin{itemize}
    \item \textbf{Interaktivní editace:} Uživatel vytváří a upravuje graf pomocí grafického editoru (viz sekce \ref{sec:GamesInteractive}).
    \item \textbf{Generování náhodných her:} Automatické vytvoření náhodného grafu podle zadaných parametrů (viz sekce \ref{sec:GamesGeneration}).
    \item \textbf{Předpřipravené sady:} Načtení ukázkových připravených příkladů(viz sekce \ref{sec:GamesPreparedSets}).
\end{itemize}

Všechny metody využívají komponentu \texttt{DisplayGraph} pro vizualizaci herního grafu (viz sekce \ref{sec:GamesVisualization}). Vnitřní struktura dat je realizována pomocí tříd \texttt{GamePosition} a \texttt{GameGraph} definovaných v modulu \texttt{NodeClasses.js}.

\section{Interaktivní editace grafu}
\label{sec:GamesInteractive}

Komponenta \texttt{ManualInput} umožňuje vytvářet a upravovat herní grafy pomocí interaktivního editoru. Uživatel může:

\begin{itemize}
    \item \textbf{Přidávat uzly:} Vytvořit novou pozici pro Hráče 1 nebo Hráče 2 pomocí tlačítek pod grafem.
    
    \item \textbf{Upravovat uzly:} Kliknutím na uzel ho označíme. Označený uzel můžeme:
    \begin{itemize}
        \item Změnit na pozici Hráče 1 nebo Hráče 2
        \item Odstranit z grafu
        \item Nastavit jako počáteční pozici
        \item Použít jako zdroj nebo cíl pro vytvoření hrany
    \end{itemize}
    
    \item \textbf{Vytvářet hrany:} Po označení prvního uzlu aktivujeme režim přidávání hrany. Následným kliknutím na druhý uzel vytvoříme hranu z prvního označeného uzlu do druhého.
    
    \item \textbf{Odstraňovat hrany:} Označíme uzel, aktivujeme režim odstranění hrany a klikneme na cílový uzel.
    
    \item \textbf{Reorganizovat graf:} Uzly lze přesouvat myší pro lepší vizuální uspořádání.
\end{itemize}

Aplikace průběžně validuje graf a zabraňuje vytvoření neplatných struktur. Pokud uživatel zkusí přidat hranu, která by vytvořila cyklus, aplikace tuto akci zablokuje a zobrazí chybovou zprávu. Dále upozorní uživatele, pokud není nastaven počáteční uzel, což je nutné pro spuštění analýzy.

\section{Generování náhodných her}
\label{sec:GamesGeneration}

Modul \texttt{Generator.js} obsahuje funkci \texttt{generateGraph()} pro vytváření náhodných herních grafů. Uživatel nastavuje dva parametry:

\begin{itemize}
    \item \textbf{Počet pozic:} Kolik uzlů (herních pozic) bude graf obsahovat.
    \item \textbf{Pravděpodobnost hrany:} Hodnota 0--100 určující, jak pravděpodobné je vytvoření hrany mezi dvěma uzly.
\end{itemize}

Algoritmus generování probíhá ve dvou fázích:

\begin{enumerate}
    \item \textbf{Vytvoření kostry:} Nejprve vytvoříme uzly očíslované od 0 do \( n-1 \), kde \( n \) je počet pozic. Každému uzlu náhodně přiřadíme Hráče 1 nebo 2. Poté pro každý uzel \( i > 0 \) vytvoříme hranu z náhodného uzlu \( j < i \) do uzlu \( i \). To zajistí, že výsledný graf je acyklický (DAG) a souvislý – z uzlu 0 lze dosáhnout všech ostatních uzlů.
    
    \item \textbf{Přidání dalších hran:} Procházíme všechny dvojice uzlů \( (i, j) \), kde \( i < j \), a s pravděpodobností odpovídající vstupu přidáme hranu \( i \to j \), pokud tato hrana ještě neexistuje. Směr je vždy z nižšího indexu k vyššímu, což zachovává acykličnost.
\end{enumerate}

Výsledný graf je vždy platný DAG s uzlem 0 jako počáteční pozicí.

\section{Vizualizace grafu}
\label{sec:GamesVisualization}

Vizualizace herního grafu využívá komponentu \texttt{DisplayGraph}, která je postavena na knihovně \emph{react-force-graph-2d}. Graf zobrazuje:

\begin{itemize}
    \item \textbf{Uzly:} Pozice jsou obarveny podle přiřazeného hráče – modrá pro Hráče 1, červená pro Hráče 2. Výherní pozice mají světlejší odstín, prohrávající tmavší.
    
    \item \textbf{Počáteční pozice:} Označena výrazným rámečkem a nápisem „START".
    
    \item \textbf{Hrany:} Možné tahy jsou zobrazeny jako šipky. Hrany patřící do optimální strategie jsou zvýrazněny tlustší čarou a jasnější barvou.
    
    \item \textbf{Popisky:} Každý uzel zobrazuje své ID a přiřazeného hráče.
\end{itemize}

Graf používá fyzikální simulaci pro automatické rozmístění uzlů. Uživatel může uzly přesouvat myší a graf přibližovat nebo oddalovat kolečkem myši.

\section{Propojení s MCVP}
\label{sec:GamesMCVPConnection}

Aplikace demonstruje P-úplnost implementací převodu z MCVP na kombinatorickou hru (viz kapitola \ref{sec:MCVPtoGame}). Tento převod je realizován v modulu \texttt{ConversionCombinatorialGame.js}.

Princip převodu mapuje hradla MCVP obvodu na herní pozice:

\begin{itemize}
    \item Hradlo OR odpovídá pozici Hráče I – Hráč I si volí, kterou větev následovat
    \item Hradlo AND odpovídá pozici Hráče II – Hráč II volí větev
    \item Proměnná s hodnotou 1 vytváří pozici Hráče II bez tahů (Hráč I vyhrává)
    \item Proměnná s hodnotou 0 vytváří pozici Hráče I bez tahů (Hráč I prohrává)
\end{itemize}

Výsledná hra zachovává sémantiku – Hráč I má výherní strategii právě tehdy, když MCVP obvod vyhodnotí na hodnotu 1.

\section{Ukládání a načítání her}
\label{sec:GamesSerialization}

Herní grafy lze exportovat a importovat ve formátu JSON pomocí komponenty \texttt{FileTransferControls}. Formát obsahuje:

\begin{itemize}
    \item Pole \texttt{nodes} s uzly – každý má ID a přiřazeného hráče
    \item Pole \texttt{edges} s hranami ve formátu source-target
    \item \texttt{startingPosition} určující ID počáteční pozice
\end{itemize}

Tento formát umožňuje sdílení her mezi uživateli a vytváření knihovny příkladů.

\subsection{Předpřipravené sady}
\label{sec:GamesPreparedSets}

Ve složce \texttt{Sady/CombinatorialGame} najdeme předpřipravené herní grafy různé velikosti a složitosti. Tyto sady slouží jako ukázkové příklady a výchozí bod pro experimentování s algoritmem.

\section{Algoritmus analýzy}
\label{sec:GamesAlgorithm}

Řešení problému kombinatorických her je implementováno v modulu \texttt{ComputeWinner.js}. Algoritmus je technicky velmi podobný vyhodnocování MCVP stromu – používá průchod do hloubky s memoizací. Rozdíl spočívá v logice, kterou určujeme výsledek každé pozice.

\subsection{Vyhodnocení výherních pozic}
\label{sec:GamesRetrograde}

Algoritmus \texttt{computeWinner()} používá průchod do hloubky (DFS) s memoizací pro určení výherních pozic. Pro každou pozici \( p \) určuje, zda je výherní pro Hráče I:

\begin{itemize}
    \item \textbf{Koncová pozice Hráče II:} Pokud je Hráč II na tahu a nemá žádné možné tahy, Hráč I vyhrává. Taková pozice je výherní.
    
    \item \textbf{Pozice Hráče I:} Hráč I vyhrává, pokud existuje alespoň jeden tah do výherní pozice. Stačí jedna výherní možnost.
    
    \item \textbf{Pozice Hráče II:} Hráč I vyhrává pouze tehdy, když všechny možné tahy Hráče II vedou do výherních pozic pro Hráče I. Jinak Hráč II může zvolit tah do prohrávající pozice a zabránit výhře Hráče I.
\end{itemize}

Formálně můžeme zapsat:

\begin{itemize}
    \item Pozice \( p \) je výherní pro Hráče I, pokud:
    \begin{itemize}
        \item \( p \) patří Hráči II a nemá následníky, nebo
        \item \( p \) patří Hráči I a existuje následník \( q \) takový, že \( q \) je výherní, nebo
        \item \( p \) patří Hráči II a všichni následníci \( q \) jsou výherní.
    \end{itemize}
\end{itemize}

Algoritmus používá zásobník pro iterativní průchod grafem a slovník \texttt{memo} pro ukládání již vyhodnocených pozic. Detekuje také cykly pomocí množiny \texttt{processing} – cyklické pozice jsou považovány za prohrávající.

Časová složitost je \( O(V + E) \), kde \( V \) je počet uzlů a \( E \) počet hran, protože každá pozice a hrana je zpracována právě jednou.

\subsection{Optimální tahy}
\label{sec:GamesOptimalMoves}

Funkce \texttt{getOptimalMoves()} identifikuje hrany, které jsou součástí výherní strategie. Hrana z pozice \( u \) do pozice \( v \) je optimální, pokud obě pozice jsou výherní pro Hráče I. Tyto hrany jsou zvýrazněny ve vizualizaci a ukazují uživateli cestu k výhře.

\section{Krokové vyhodnocení}
\label{sec:GamesStepByStep}

Pro vzdělávací účely implementuje komponenta \texttt{StepByStepGame} krokovatelnou analýzu. Funkce \texttt{computeWinningStrategySteps()} zaznamenává každý krok vyhodnocení:

\begin{itemize}
    \item ID analyzované pozice
    \item Výsledky vyhodnocení všech potomků
    \item Finální rozhodnutí, zda je pozice výherní
    \item Textové vysvětlení rozhodnutí
\end{itemize}

Uživatel může procházet kroky analýzy pomocí navigačních tlačítek. Aktuálně analyzovaný uzel je ve vizualizaci zvýrazněn a vedle grafu se zobrazuje detailní vysvětlení logiky rozhodnutí. Tato funkce pomáhá pochopit, jak algoritmus postupně buduje znalost o výherních pozicích od listů ke kořeni.
