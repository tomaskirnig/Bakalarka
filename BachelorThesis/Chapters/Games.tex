\chapter{Kombinatorická hra}
\label{sec:Games}

\section{Teoretický základ}
\label{sec:GamesTheory}

Kombinatorická hra dvou hráčů na orientovaném grafu je další příklad P-úplného problému \cite{sawa}. Hra má tyto vlastnosti:

\begin{itemize}
    \item \textbf{Dva hráči:} Každé pole grafu má specifikováno, který hráč je na tahu – Hráč I (první hráč) nebo Hráč II (druhý hráč).
    \item \textbf{Konečná pozice:} Hra končí, když je hráč na tahu v pozici bez možných dalších tahů.
\end{itemize}

Problém spočívá v rozhodnutí, zda Hráč I má výherní strategii ze zadané počáteční pozice:

\begin{itemize}
    \item \textbf{Vstup:} Orientovaný acyklický graf (DAG), kde každý uzel reprezentuje herní pozici přiřazenou některému z hráčů, hrany reprezentují možné tahy a jeden uzel je označen jako počáteční pozice.
    \item \textbf{Výstup:} Rozhodnutí, zda Hráč I má výherní strategii ze startovní pozice.
\end{itemize}

Hráč I vyhrává, pokud se dostane do pozice, kde Hráč II nemá žádný možný tah. Naopak, Hráč I prohrává, když se sám dostane do pozice bez možných tahů \cite{sawa}.

\subsection{P-úplnost problému}
\label{sec:GamesCompleteness}

Tento problém je P-úplný \cite{sawa, miyano}. Určení výherní strategie lze provést v polynomiálním čase pomocí tzv. retrográdní analýzy \cite{sawa}, která zpětně vyhodnocuje pozice od koncových uzlů. Implementace je podobná algoritmu vyhodnocování MCVP (viz kapitola \ref{sec:MCVPEvaluation}) – opět používáme průchod do hloubky s memoizací. Liší se pouze logika rozhodování: místo kombinace logických hodnot zde určujeme, která pozice je výherní na základě možností volby jednotlivých hráčů. I když je problém řešitelný v třídě P, jeho P-úplnost naznačuje, že pravděpodobně neexistuje efektivní paralelní algoritmus – řešení vyžaduje sekvenční zpracování pozic.

\section{Formát vstupu}
\label{sec:GamesInput}

Aplikace nabízí tři způsoby zadávání herního grafu:

\begin{itemize}
    \item \textbf{Interaktivní editace:} Uživatel vytváří a upravuje graf pomocí grafického editoru (viz sekce \ref{sec:GamesInteractive}).
    \item \textbf{Generování náhodných her:} Automatické vytvoření náhodného grafu podle zadaných parametrů (viz sekce \ref{sec:GamesGeneration}).
    \item \textbf{Předpřipravené sady:} Načtení ukázkových připravených příkladů(viz sekce \ref{sec:GamesPreparedSets}).
\end{itemize}

Všechny metody využívají komponentu \texttt{DisplayGraph} pro vizualizaci herního grafu (viz sekce \ref{sec:GamesVisualization}).  

\section{Interaktivní editace grafu}
\label{sec:GamesInteractive}

Komponenta \texttt{ManualInput} umožňuje vytvářet a upravovat herní grafy pomocí interaktivního grafického editoru. Uživatel může:

\begin{itemize}
    \item \textbf{Přidávat uzly:} Vytvořit novou pozici.
    
    \item \textbf{Upravovat uzly:} Kliknutím na uzel ho označíme. Označený uzel můžeme:
    \begin{itemize}
        \item Změnit hráče který je na tahu (Hráč I nebo Hráč II)
        \item Odstranit pozici z grafu
        \item Nastavit jako počáteční pozici
        \item Použít jako zdroj nebo cíl pro vytvoření hrany
        \item Vytvořit hranu z nebo do tohoto uzlu
        \item Odstranit hrany k nebo od tohoto uzlu
    \end{itemize}
            
    \item \textbf{Reorganizovat graf:} Uzly lze přesouvat myší pro lepší vizuální uspořádání.
\end{itemize}

Aplikace průběžně validuje graf a zabraňuje vytvoření neplatných struktur. Pokud uživatel zkusí přidat hranu, která by vytvořila cyklus, aplikace tuto akci zablokuje a zobrazí chybovou zprávu. Dále upozorní uživatele, pokud není nastavena počáteční pozice, což je nutné pro spuštění analýzy.

\section{Generování náhodných her}
\label{sec:GamesGeneration}

Modul \texttt{Generator.js} obsahuje funkci \texttt{generateGraph()} pro vytváření náhodných herních grafů. Uživatel nastavuje dva parametry:

\begin{itemize}
    \item \textbf{Počet pozic:} Kolik uzlů (herních pozic) bude graf obsahovat.
    \item \textbf{Pravděpodobnost hrany:} Hodnota 0\% -- 100\% určující, jak pravděpodobné je vytvoření hrany mezi dvěma uzly.
\end{itemize}

Algoritmus generování probíhá ve dvou fázích:

\begin{enumerate}
    \item \textbf{Vytvoření kostry:} Vytvoříme uzly očíslované 0 až \( n-1 \) a každému náhodně přiřadíme hráče. Pro každý uzel \( i > 0 \) pak vytvoříme hranu z náhodného předchozího uzlu do uzlu \( i \). Tím zajistíme souvislý acyklický graf – hrany vedou vždy od nižších čísel k vyšším, takže cyklus vzniknout nemůže.
    
    \item \textbf{Přidání dalších hran:} Pro každou dvojici uzlů \( i, j \), kde \( i < j \), podle dané pravděpodobnosti přidáme hranu \( i \to j \), pokud ještě neexistuje.
\end{enumerate}

Výsledný graf je vždy platný DAG s uzlem 0 jako počáteční pozicí.

\section{Předpřipravené sady}
\label{sec:GamesPreparedSets}

Ve složce \texttt{Sady/CombinatorialGame} najdeme předpřipravené herní grafy různé velikosti a složitosti.

\subsection{Ukládání a načítání her}
\label{sec:GamesSerialization}

Herní grafy lze exportovat a importovat ve formátu JSON pomocí komponenty \texttt{FileTransferControls}. Formát obsahuje:

\begin{itemize}
    \item Pole \texttt{nodes} s uzly – každý má ID a přiřazeného hráče
    \item Pole \texttt{edges} s hranami ve formátu source-target
    \item \texttt{startingPosition} určující ID počáteční pozice
\end{itemize}

Tento formát umožňuje sdílení her mezi uživateli a vytváření knihovny předpřipravených příkladů.

\section{Vizualizace grafu}
\label{sec:GamesVisualization}

Vizualizace herního grafu využívá komponentu \texttt{DisplayGraph}, která využívá knihovnu \emph{react-force-graph-2d}. Graf zobrazuje:

\begin{itemize}
    \item \textbf{Uzly:} Každý uzel reprezentuje herní pozici. Pod uzlem je zobrazen hráč, který je na tahu (I nebo II). Při najetí myší na libovolný uzel se v centrech všech uzlů zobrazí jejich ID pro snadnější orientaci.
    
    \item \textbf{Počáteční pozice:} Označena oranžovou barvou.
    
    \item \textbf{Hrany:} Možné tahy jsou zobrazeny jako směrované hrany mezi uzly. Hrany patřící do výherní strategie jsou zvýrazněny tlustší čarou a žlutou barvou.
\end{itemize}

Graf používá fyzikální simulaci pro automatické rozmístění uzlů. Uživatel může uzly přesouvat myší a graf přibližovat nebo oddalovat kolečkem myši.

\section{Algoritmus analýzy}
\label{sec:GamesAlgorithm}

Řešení problému kombinatorických her je implementováno v modulu \texttt{ComputeWinner.js}. Algoritmus je technicky velmi podobný vyhodnocování MCVP stromu – používá průchod do hloubky (DFS) s memoizací. Rozdíl spočívá v logice, kterou určujeme výsledek každé pozice.

\subsection{Vyhodnocení výherních pozic}
\label{sec:GamesRetrograde}

Algoritmus rozhoduje, zda je daná pozice výherní pro Hráče I. Koncové pozice (listové uzly) bez dalších tahů jsou výherní pro Hráče I, když je na tahu Hráč II (nemá možnost pokračovat). U ostatních pozic záleží na tom, kdo je na tahu: pokud je to Hráč I, stačí mu jedna výherní možnost pokračování – může si ji vybrat. Pokud je na tahu Hráč II, musí všechny možné tahy vést do výherních pozic Hráče I, protože Hráč II si vybere nejlepší tah pro sebe.

Časová složitost je \( O(V + E) \), kde \( V \) je počet uzlů a \( E \) počet hran, protože každá pozice a hrana je zpracována právě jednou.

\subsection{Optimální tahy}
\label{sec:GamesOptimalMoves}

Funkce \texttt{getOptimalMoves()} identifikuje hrany, které jsou součástí výherní strategie. Hrana z pozice \( u \) do pozice \( v \) je optimální, pokud obě pozice jsou výherní pro Hráče I. Tyto hrany jsou zvýrazněny ve vizualizaci a ukazují uživateli cestu k výhře.

\section{Krokové vyhodnocení}
\label{sec:GamesStepByStep}

Pro detailnější průchod grafem implementuje komponenta \texttt{StepByStepGame} krokovatelnou analýzu. Funkce \texttt{computeWinningStrategySteps()} zaznamenává každý krok v celkovém vyhodnocení:

\begin{itemize}
    \item Hráč na tahu v dané pozici
    \item Výsledky vyhodnocení všech potomků
    \item Finální rozhodnutí, zda je pozice výherní pro Hráče I
    \item Textové vysvětlení rozhodnutí
\end{itemize}

Uživatel může procházet kroky analýzy pomocí navigačních tlačítek – kromě standardních tlačítek „Předchozí" a „Další" jsou k dispozici také tlačítka „Začátek" a „Konec" pro rychlý přesun na první nebo poslední krok analýzy. Aktuálně analyzovaný uzel je ve zvýrazněn světle modrou barvou (stejnou jako při najetí myší) a pod grafem jse slovní popis rozhodnutí. Tato funkce pomáhá pochopit, jak algoritmus postupně buduje cesty z výherních pozicic od listů ke kořeni.
