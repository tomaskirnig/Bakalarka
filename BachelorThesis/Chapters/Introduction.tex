\chapter{Úvod}
\label{sec:Introduction}

Teoretická informatika poskytuje způsob jak zkoumat, porozumět a optimalizovat možnosti a limity výpočetních systémů. Hlavním bodem tohoto zkoumání je teorie složitosti, která klasifikuje problémy na základě zdrojů potřebných pro jejich vyřešení, jako je čas a paměť. Jednou z nejvýznamnějších složitostních tříd je třída \( P \), zahrnující problémy řešitelné v polynomiálním čase – tedy takové, jejichž časová složitost lze vyjádřit jako \( O(n^k) \) pro nějakou konstantu \( k \) (například \( k = 2 \) pro kvadratickou složitost, \( k = 3 \) pro kubickou), kde \( n \) je velikost vstupu – na deterministickém Turingově stroji \cite{papadimitriou1993}. Ačkoliv jsou problémy v této třídě tradičně považovány za „efektivně řešitelné", existují mezi nimi zásadní rozdíly, hlavně jestli uvažujeme o možnostech jejich paralelizace.

V tomto kontextu hraje klíčovou roli koncept \emph{P-úplnosti} (P-completeness). P-úplné problémy představují výpočetně nejnáročnější úlohy v rámci třídy \( P \). Jsou to problémy, na které lze v logaritmickém prostoru redukovat jakýkoliv jiný problém z \( P \). Důležitost této klasifikace spočívá v hypotéze, že P-úplné problémy pravděpodobně nelze efektivně paralelizovat (nepatří do třídy \( NC \)), a jejich řešení je tedy inherentně sekvenční \cite{arora2009}. Studium těchto problémů tak umožňuje lépe vymezit hranici mezi úlohami, které lze urychlit paralelním zpracováním, a těmi, které vyžadují striktně sériový postup.

Výuka těchto abstraktních konceptů však představuje značnou výzvu. Porozumění formálním definicím redukcí a složitostních tříd pouze prostřednictvím statického textu může být pro studenty obtížné. Zatímco pro základní algoritmy existuje řada vizualizačních nástrojů, oblast složitostní teorie – a konkrétně P-úplnost – zůstává v edukačním softwaru často opomíjena. Cílem této práce je tuto mezeru zaplnit návrhem a implementací interaktivní výukové komponenty, která umožní dynamickou demonstraci těchto principů.

Jako referenční problém byl zvolen \emph{Monotone Circuit Value Problem} (MCVP), který spočívá ve vyhodnocení logického obvodu složeného z monotonních hradel a přirozeně modeluje tok výpočtu \cite{miyano1990}. Pro demonstraci univerzálnosti konceptu P-úplnosti aplikace implementuje také simulace dvou dalších problémů z odlišných domén:
\begin{itemize}
    \item \textbf{Kombinatorické hry na grafu:} Úloha analyzující existenci vítězné strategie v deterministických hrách dvou hráčů.
    \item \textbf{Prázdnost bezkontextových gramatik:} Problém rozhodující, zda daná gramatika generuje neprázdný jazyk \cite{sawa}.
\end{itemize}

Hlavním přínosem vytvořené aplikace je možnost vizualizace nejen samotného řešení těchto úloh, ale především \emph{převodů} (redukcí) mezi nimi. Uživatel může sledovat krokovou transformaci instance MCVP na instanci hry nebo gramatiky, což názorně ilustruje princip polynomiálních redukcí a vzájemnou převoditelnost P-úplných problémů \cite{arora2009}.

Výsledkem práce je moderní webová aplikace navržená jako flexibilní nástroj pro výuku. Umožňuje uživatelům pracovat s vlastními vstupy, generovat náhodné instance pro testování a využívat předpřipravené sady úloh pro řízené studium.

Text práce je členěn do logických celků. Po úvodu následuje kapitola věnovaná použitým technologiím a architektuře aplikace. Jádro práce tvoří tři kapitoly, z nichž každá se detailně věnuje jednomu z implementovaných problémů: nejprve Monotone Circuit Value Problem (MCVP), následně kombinatorické hry na grafu a nakonec problém prázdnosti bezkontextových gramatik. Závěr práce shrnuje dosažené výsledky a navrhuje možnosti dalšího rozšíření.

\endinput