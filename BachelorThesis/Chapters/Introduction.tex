\chapter{Úvod}
\label{sec:Introduction}

Teoretická informatika poskytuje způsob, jak zkoumat, porozumět a optimalizovat možnosti a limity výpočetních systémů. Hlavním bodem tohoto zkoumání je teorie složitosti, která klasifikuje problémy na základě zdrojů potřebných pro jejich vyřešení, jako je čas a paměť. Jednou z nejvýznamnějších složitostních tříd je třída \( P \), zahrnující problémy řešitelné v polynomiálním čase na deterministickém Turingově stroji \cite{uti_turing} – tedy takové, jejichž časovou složitost lze vyjádřit jako \( O(n^k) \) \cite{sawa_bigo} pro nějakou konstantu \( k \), kde \( n \) je velikost vstupu. Přestože jsou problémy v této třídě obvykle pokládány za \uv{efektivně řešitelné}, projevují se mezi nimi významné odlišnosti, zejména pokud začneme řešit jejich paralelizovatelnost.

V tomto kontextu hraje klíčovou roli podmnožina \emph{P-úplných} problémů (\emph{P-complete} problems). P-úplné problémy představují výpočetně nejnáročnější úlohy v rámci třídy \( P \). Jsou to problémy, na které lze s logaritmickou paměťovou složitostí převést jakýkoliv jiný problém z třídy \( P \) \cite{papadimitriou, arora}. Hlavní problém u této podmnožiny spočívá v předpokladu, že P-úplné problémy pravděpodobně nelze efektivně paralelizovat. To znamená, že tyto úlohy nespadají do třídy \( NC \), která obsahuje problémy řešitelné v polylogaritmickém čase \( O(\log^k n) \) pomocí paralelního výpočtu s polynomiálně mnoha procesory \cite{sawa}. Z toho vyplývá, že na rozdíl od NC problémů řešení P-úplných problémů vyžaduje sekvenční zpracování. Studium P-úplnosti nám tak pomáhá vymezit hranici mezi paralelizovatelným a čistě sekvenčním výpočtem.

Tento projekt představuje interaktivní ukázku konceptu P-úplnosti prostřednictvím webové aplikace. Cílem je vytvořit nástroj, který umožní uživatelům vizualizovat a pochopit výpočet a převod mezi zvolenými P-úplnými problémy.

Jako hlavní problém byl zvolen \emph{Monotone Circuit Value Problem} (MCVP), ve kterém se vyhodnocuje logický obvod tvořený pouze hradly typu AND a OR \cite{sawa}. Pro demonstraci univerzálnosti konceptu P-úplnosti aplikace implementuje také simulace dvou dalších problémů:
\begin{itemize}
    \item \textbf{Kombinatorické hry na grafu:} Úloha analyzující existenci vítězné strategie v deterministické hře dvou hráčů \cite{sawa}.
    \item \textbf{Problém neprázdnosti jazyka bezkontextové gramatiky:} Rozhoduje, zda daná gramatika generuje neprázdný jazyk \cite{sawa}.
\end{itemize}

Hlavním přínosem vytvořené aplikace je možnost vizualizace nejen samotného řešení těchto úloh, ale především \emph{převodů} (redukcí) mezi nimi. Uživatel může sledovat krokovou transformaci instance MCVP na instanci hry nebo gramatiky, což názorně ilustruje princip polynomiálních redukcí a vzájemnou převoditelnost P-úplných problémů \cite{sawa}.

Výsledkem práce je webová aplikace navržená jako flexibilní nástroj pro výuku. Umožňuje uživatelům pracovat s vlastními vstupy, generovat náhodné instance pro testování a využívat předpřipravené sady úloh.

Text práce je členěn do několika částí. Po úvodu následuje kapitola věnovaná teoretickému základu, definicím klíčových pojmů a principům redukcí. Další část se zabývá analýzou a návrhem aplikace, volbou technologií a popisem datových struktur. Stěžejní částí práce je kapitola věnovaná detailnímu popisu implementace všech tří problémů a jejich vzájemných převodů. Závěr práce shrnuje dosažené výsledky a navrhuje možnosti dalšího rozšíření.

\endinput