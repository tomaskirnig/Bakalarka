\chapter{Úvod}
\label{sec:Introduction}

Teoretická informatika tvoří matematický základ pro pochopení možností a limitů výpočetních systémů. V jejím jádru leží teorie složitosti, která klasifikuje problémy na základě zdrojů potřebných k jejich vyřešení, jako je čas a paměť. Jednou z nejznámějších a nejdůležitějších složitostních tříd je třída \( P \), zahrnující problémy, které lze vyřešit v polynomiálním čase na deterministickém Turingově stroji \cite{papadimitriou1993}. Ačkoliv jsou problémy v třídě \( P \) v kontextu teoretické informatiky často označovány za „efektivně řešitelné“, praxe ukazuje, že ne všechny jsou si rovny, zejména pokud uvažujeme o možnostech jejich paralelizace.

Zde vstupuje do popředí koncept \emph{P-úplnosti} (P-completeness). P-úplné problémy představují ty nejtěžší úlohy uvnitř třídy \( P \). Jsou to problémy, na které lze v logaritmickém prostoru redukovat jakýkoliv jiný problém z \( P \). Význam této klasifikace spočívá v hypotéze, že P-úplné problémy pravděpodobně nelze efektivně paralelizovat (tj. nepatří do třídy \( NC \)), a jejich řešení je tedy inherentně sekvenční \cite{arora2009}. Studium těchto problémů nám tak pomáhá porozumět hranicím mezi tím, co můžeme urychlit přidáním výpočetního výkonu, a tím, kde musíme postupovat krok za krokem.

Výuka těchto konceptů však naráží na značné překážky. Abstrakce redukcí mezi problémy a formální definice složitostních tříd jsou pro studenty často obtížně uchopitelné pouze prostřednictvím statických textů či diagramů. Zatímco pro základní algoritmy (třídění, vyhledávání) existuje řada vizualizačních nástrojů, oblast složitosti – a konkrétně P-úplnost – zůstává v edukačním softwaru často opomíjena. Cílem této práce je tuto mezeru zaplnit a poskytnout interaktivní nástroj, který převede abstraktní definice do dynamické a vizuální podoby.

Jako ústřední problém pro tuto práci byl zvolen \emph{Monotone Circuit Value Problem} (MCVP). Jedná se o kanonický P-úplný problém, který spočívá ve vyhodnocení logického obvodu složeného z monotonních hradel (AND, OR) \cite{miyano1990}. MCVP je ideálním kandidátem pro výuku, protože přirozeně modeluje tok výpočtu. Aplikace vyvinutá v rámci této práce však nezůstává pouze u MCVP. Abychom demonstrovali univerzálnost konceptu P-úplnosti, implementujeme také simulace dvou dalších problémů z odlišných domén:
\begin{itemize}
    \item \textbf{Kombinatorické hry na grafu:} Úloha, která analyzuje vítězné strategie v deterministických hrách dvou hráčů, což propojuje teorii složitosti s teorií her.
    \item \textbf{Prázdnost bezkontextových gramatik:} Problém z oblasti formálních jazyků, zjišťující, zda daná gramatika generuje alespoň jedno slovo složené z terminálních symbolů \cite{sawa}.
\end{itemize}

Klíčovým přínosem vytvořené aplikace není pouze možnost tyto problémy izolovaně řešit, ale především schopnost vizualizovat \emph{převody} (redukce) mezi nimi. Uživatel může sledovat, jak se instance MCVP transformuje na instanci hry nebo gramatiky, což názorně demonstruje princip, že vyřešení jednoho P-úplného problému umožňuje vyřešit jakýkoliv jiný. Tento proces je doprovázen vizualizací změn stavu a slovním popisem, což studentům umožňuje nahlédnout "pod kapotu" polynomiálních redukcí \cite{arora2009}.

Výsledkem práce je moderní webová aplikace, která slouží jako flexibilní výuková komponenta. Umožňuje uživatelům experimentovat s vlastními vstupy, generovat náhodné instance pro testování hypotéz a využívat předpřipravené sady úloh pro řízenou výuku.

Text práce je členěn do několika logických celků. Po úvodu následuje kapitola věnovaná použitým technologiím a architektuře aplikace. Jádro práce tvoří tři kapitoly, z nichž každá se detailně věnuje jednomu z implementovaných problémů: nejprve Monotone Circuit Value Problem (MCVP), následně kombinatorické hry na grafu a nakonec problém prázdnosti bezkontextových gramatik. Závěr práce shrnuje dosažené výsledky a navrhuje možnosti dalšího rozšíření.

\endinput