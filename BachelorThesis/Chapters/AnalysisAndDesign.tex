\chapter{Analýza a návrh}
\label{sec:AnalysisAndDesign}

Tato kapitola popisuje zvolené technologie a architekturu aplikace. Dále představuje datové struktury pro reprezentaci jednotlivých problémů a principy návrhu uživatelského rozhraní.

\section{Volba technologií}
\label{sec:WebTechnologies}

Aplikace byla implementována jako webová, protože nevyžaduje instalaci a běží v libovolném moderním prohlížeči. Webové řešení také umožňuje snadnou údržbu a okamžitou distribuci aktualizací.

\subsection{React a Vite}
\label{sec:ReactVite}

Pro vývoj byla použita knihovna \emph{React} \cite{react2024}. React umožňuje rozdělit uživatelské rozhraní na znovupoužitelné komponenty, což usnadnilo vývoj modulů pro jednotlivé problémy. Díky reaktivitě se rozhraní automaticky synchronizuje se stavem dat bez nutnosti ruční manipulace s DOM.

Pro sestavení aplikace byl použit nástroj \emph{Vite} \cite{vite2024}. Vite nabízí rychlé vývojové prostředí s okamžitou odezvou při úpravách kódu a optimalizuje soubory pro produkční verzi.

\section{Architektura a struktura kódu}
\label{sec:ComponentArchitecture}

Aplikace je navržena jako \emph{Single Page Application} (SPA) – veškerá logika a navigace probíhá v rámci jedné stránky, podobně jako u desktopové aplikace. Kořenová komponenta \texttt{App.jsx} spravuje globální stav a přepínání mezi moduly.

Kód je strukturován tak, aby logika byla oddělena od vizualizace. Každý modul (MCVP, hry, gramatiky) disponuje vlastní adresářovou strukturou:
\begin{itemize}
    \item \textbf{Hlavní komponenta:} Funguje jako kontejner, který spravuje lokální stav a propojuje vstupní data s algoritmy a vizualizací.
    \item \textbf{Adresář Utils:} Zde jsou soustředěny čisté algoritmy – parsery pro zpracování vstupu, generátory náhodných instancí a evaluátory pro výpočet řešení.
    \item \textbf{Vizualizační komponenty:} Starají se o vykreslení grafů a stromů pomocí knihovny \emph{react-force-graph}.
\end{itemize}

\section{Návrh datových struktur}
\label{sec:DataStructures}

Pro práci s P-úplnými problémy bylo nutné navrhnout jejich vnitřní reprezentaci v paměti.

\subsection{Reprezentace obvodu MCVP}
\label{sec:MCVPDesignStructure}

Pro modelování logického obvodu byl využit orientovaný acyklický graf. Každý uzel obvodu je reprezentován instancí třídy \texttt{Node}. Tato třída, jak ukazuje třídní diagram na obrázku \ref{fig:mcvp-class}, uchovává informaci o typu operace (AND, OR) nebo hodnotě proměnné. Seznam odkazů na potomky umožňuje rekurzivní průchod stromem při vyhodnocování.

\begin{figure}[htbp]
    \centering
    \includegraphics[width=0.4\textwidth]{IMGs/MCVP_class.png}
    \caption{Třídní diagram pro hlavní třídy MCVP}
    \label{fig:mcvp-class}
\end{figure}

\subsection{Reprezentace herního grafu}
\label{sec:GamesDesignStructure}

Na rozdíl od MCVP může herní graf obsahovat cykly, proto byly navrženy třídy \texttt{GamePosition} a \texttt{GameGraph} (viz obrázek \ref{fig:cg-class}). Třída \texttt{GamePosition} reprezentuje jednotlivou pozici a uchovává informaci o hráči na tahu spolu se seznamy předchůdců a následníků, což je nezbytné pro retrográdní analýzu.

\begin{figure}[htbp]
    \centering
    \includegraphics[width=0.8\textwidth]{IMGs/CG_class.png}
    \caption{Třídní diagram pro hlavní třídy kombinatorické hry}
    \label{fig:cg-class}
\end{figure}

\subsection{Reprezentace bezkontextové gramatiky}
\label{sec:GrammarDesignStructure}

U bezkontextových gramatik je zaměřeno na reprezentaci přepisovacích pravidel. Třída \texttt{Grammar} (obrázek \ref{fig:grammar-class}) spravuje množiny symbolů a seznamy produkcí. Každá produkce propojuje levou stranu (neterminál) s polem symbolů na pravé straně. Tato struktura umožňuje snadno identifikovat produktivní neterminály.

\begin{figure}[htbp]
    \centering
    \includegraphics[width=0.3\textwidth]{IMGs/Grammar_class.png}
    \caption{Třídní diagram pro hlavní třídu bezkontextové gramatiky}
    \label{fig:grammar-class}
\end{figure}

\section{Návrh uživatelského rozhraní}
\label{sec:UIDesign}

Rozhraní je navrženo tak, aby student nebyl zahlcen informacemi a mohl se soustředit na podstatu problému. Rozložení stránky bylo sjednoceno napříč všemi moduly:
\begin{itemize}
    \item \textbf{Ovládací panel:} Levá nebo horní část obrazovky, kde uživatel volí způsob zadání vstupu a spouští simulace.
    \item \textbf{Interaktivní plátno:} Hlavní plocha pro vizualizaci grafů, kde uživatel může s prvky přímo manipulovat.
    \item \textbf{Informační panely:} Modální okna nebo postranní panely zobrazující detaily o probíhajícím výpočtu nebo krokovém převodu.
\end{itemize}

Pro vizualizaci byla použita knihovna \emph{react-force-graph} \cite{reactforcegraph2024}, která pomocí fyzikální simulace automaticky rozmísťuje uzly a hrany. Stylování je realizováno frameworkem \emph{Bootstrap 5} \cite{bootstrap2024}, který zajišťuje jednotný vzhled a responzivitu na různých velikostech obrazovek.