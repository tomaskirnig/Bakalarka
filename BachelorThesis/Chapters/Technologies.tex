\chapter{Použité technologie a architektura}
\label{sec:Technologies}

Tato kapitola popisuje technologický základ vytvořené aplikace a její architekturu. Výběr technologií byl veden požadavky na moderní, interaktivní a snadno rozšiřitelné řešení vhodné pro výukové účely.

\section{Webové technologie}
\label{sec:WebTechnologies}

Pro implementaci výukové aplikace jsme zvolili webové technologie, které nabízejí řadu výhod oproti desktopovým nebo mobilním aplikacím. Webová aplikace nevyžaduje instalaci a běží v libovolném moderním webovém prohlížeči, což zajišťuje maximální dostupnost pro uživatele napříč různými platformami a operačními systémy. Dalším přínosem je snadná údržba – aktualizace aplikace se projeví okamžitě u všech uživatelů bez nutnosti distribuce nových verzí.

\section{React a Vite}
\label{sec:ReactVite}

Jako hlavní framework pro vývoj uživatelského rozhraní byl zvolen \emph{React} \cite{react2024}. React je moderní JavaScriptová knihovna vyvinutá společností Meta (dříve Facebook), která umožňuje vytvářet interaktivní uživatelská rozhraní na bázi komponent. Hlavní výhodou Reactu je koncept \emph{reaktivity} – uživatelské rozhraní se automaticky aktualizuje při změně dat bez nutnosti manuální manipulace s DOM (Document Object Model).

React využívá deklarativní přístup k tvorbě UI. Na rozdíl od tradičního imperativního programování, kde vývojář musí krok za krokem instruovat prohlížeč, jak upravit rozhraní, v Reactu stačí popsat, jak má výsledné rozhraní vypadat, a React se postará o potřebné změny v DOM. Tento přístup výrazně zjednodušuje vývoj komplexnějších aplikací a minimalizuje chyby spojené s nekonzistentním stavem.

Pro sestavení a vývoj aplikace jsme použili \emph{Vite} \cite{vite2024}. Vite je moderní \emph{bundler} – tedy nástroj, který spojuje všechny zdrojové soubory aplikace (JavaScript, CSS, obrázky) do optimalizované podoby připravené pro odeslání do prohlížeče. Během vývoje Vite nabízí významnou výhodu: místo zdlouhavého spojování všech souborů do jednoho velkého balíku servíruje jednotlivé moduly přímo do prohlížeče v jejich původní podobě. To výrazně urychluje spouštění vývojového serveru a umožňuje okamžité promítnutí změn v kódu (tzv. hot module replacement). Pro produkční nasazení pak Vite vytvoří klasický optimalizovaný build, kde jsou všechny soubory sloučeny, zmenšeny a připraveny pro rychlé načítání.

\section{Správa stavu a komponentová architektura}
\label{sec:ComponentArchitecture}

Aplikace je strukturována jako \emph{Single Page Application} (SPA), kde celá aplikace běží v rámci jedné HTML stránky a navigace mezi jednotlivými moduly probíhá bez opětovného načítání stránky. Hlavní komponenta \texttt{App.jsx} funguje jako kořen aplikační struktury a spravuje globální stav aplikace pomocí React Hooks, především \texttt{useState} pro udržení aktuální stránky a předávaných dat.

Architektura je založena na principu \emph{unidirectional data flow} (jednosměrný tok dat), kdy data tečou z rodičovské komponenty do potomků prostřednictvím properties (props), zatímco akce a události se propagují zpět nahoru pomocí callback funkcí. Tento přístup zajišťuje předvídatelné chování aplikace a usnadňuje ladění.

Aplikace je rozdělena do logických modulů podle jednotlivých problémů (viz kapitoly \ref{sec:MCVP}, \ref{sec:Games} a \ref{sec:Grammars}). Každý modul je organizován do vlastní složky obsahující:
\begin{itemize}
    \item \textbf{Hlavní komponentu} – kontejnerovou komponentu řídící celý modul
    \item \textbf{Utils} – pomocné funkce obsahující algoritmy (parsery, generátory, evaluátory)
    \item \textbf{InputSelectionComponents} – komponenty pro různé způsoby zadání vstupu
    \item \textbf{Vizualizační komponenty} – komponenty pro grafické zobrazení problémů a jejich řešení
\end{itemize}

Pro sdílení společné funkcionality mezi moduly byly vytvořeny \emph{custom hooks} (vlastní React Hooks) v adresáři \texttt{src/Hooks}. Hook \texttt{useGraphColors} centralizuje správu barev používaných ve vizualizacích a načítá je z CSS proměnných, což umožňuje snadnou změnu barevného schématu. Hook \texttt{useGraphSettings} poskytuje konfigurační parametry pro force-directed grafy, jako jsou poloměry uzlů, síly odpudivosti a vzdálenosti hran.

\section{Vizualizace grafů}
\label{sec:GraphVisualization}

Pro vizualizaci stromových a grafových struktur používá aplikace knihovnu \emph{react-force-graph-2d} \cite{reactforcegraph2024}, která je React wrapperem nad výkonnou knihovnou \emph{D3.js} (Data-Driven Documents) \cite{d3js2024}. D3.js je de facto standardem pro tvorbu datových vizualizací ve webovém prostředí a poskytuje rozsáhlé možnosti pro manipulaci s DOM na základě dat.

\subsection{Force-directed layout}
\label{sec:ForceDirectedLayout}

Klíčovým konceptem využívaným pro rozmístění uzlů v grafu je \emph{force-directed layout} (rozložení řízené silami). Tento algoritmus modeluje graf jako fyzikální systém, kde uzly představují nabitá tělesa odpudící se navzájem a hrany fungují jako pružiny přitahující spojené uzly k sobě. Simulace běží iterativně, dokud systém nedosáhne stavu s minimální energií, tedy vizuálně příjemného a čitelného rozložení.

Konkrétně aplikace využívá následující síly:
\begin{itemize}
    \item \textbf{Link force} – síla pružin mezi spojenými uzly, udržuje požadovanou vzdálenost hran
    \item \textbf{Charge force} – odpudivá síla mezi všemi uzly, zabraňuje jejich překrývání
    \item \textbf{Collision force} – detekuje a řeší kolize mezi uzly na základě jejich poloměrů
    \item \textbf{Center force} – udržuje celý graf vycentrovaný v zobrazovací ploše
\end{itemize}

Pro stromové struktury v MCVP modulu (viz kapitola \ref{sec:MCVP}) je aktivován speciální \emph{DAG mód} (Directed Acyclic Graph), který automaticky organizuje uzly do hierarchických úrovní shora dolů (\texttt{dagMode="td"}). To zajišťuje, že výraz je vizualizován tak, jak je běžné v teoretické informatice – kořen nahoře, listy dole.

\subsection{Vlastní vykreslování}
\label{sec:CustomRendering}

Knihovna react-force-graph-2d umožňuje definovat vlastní vykreslovací funkce (\texttt{nodeCanvasObject}, \texttt{linkCanvasObjectMode}), což aplikace využívá pro vizualizaci stavů uzlů a hran během evaluace. Například v MCVP modulu se uzly vybarvují podle jejich typu (proměnné vs. operátory) a vyhodnoceného stavu (0 nebo 1), zatímco hrany mění barvu podle toho, zda již byly zpracovány v průchodu stromem.

Vykreslování probíhá na HTML5 Canvas elementu, který nabízí vysoký výkon i pro grafy s desítkami či stovkami uzlů. Oproti SVG přístupu je Canvas renderování efektivnější pro časté překreslování, což je klíčové pro animace krokování algoritmy.

\section{Stylování a responzivita}
\label{sec:Styling}

Pro stylování uživatelského rozhraní aplikace využívá framework \emph{Bootstrap 5} \cite{bootstrap2024}. Bootstrap poskytuje předpřipravené CSS třídy pro tvorbu responzivního layoutu, komponent jako jsou tlačítka, formuláře, modální okna a navigační prvky. Použití Bootstrapu výrazně urychlilo vývoj a zajistilo konzistentní vzhled napříč celou aplikací.

Responzivita je implementována pomocí Bootstrapového grid systému založeného na Flexboxu. Layout se automaticky přizpůsobuje velikosti obrazovky, přičemž na mobilních zařízeních se navigace transformuje do off-canvas menu, které se vysouvá ze strany obrazovky. Toto řešení zajišťuje dobrou použitelnost aplikace i na tabletech a smartphonech.

Vedle Bootstrapu jsou specifické styly definovány v centrálním souboru \texttt{style.css}, který využívá CSS custom properties (proměnné) pro správu barevného schématu. Tento přístup umožňuje snadnou změnu celého vzhledu aplikace změnou několika proměnných.

\section{Správa vstupů a výstupů}
\label{sec:InputOutput}

Každý modul aplikace implementuje tři základní způsoby zadání vstupu:
\begin{enumerate}
    \item \textbf{Manuální zadání} – uživatel zadává vstup prostřednictvím textového pole (např. výraz v notaci pro MCVP) nebo interaktivního grafického editoru
    \item \textbf{Náhodné generování} – algoritmus generuje náhodnou instanci problému na základě parametrů zadaných uživatelem
    \item \textbf{Připravené sady} – výběr z předpřipravených příkladů uložených ve formátu JSON v adresáři \texttt{Sady/}
\end{enumerate}

Pro import a export dat aplikace používá formát JSON, který je standardním formátem pro výměnu dat ve webových aplikacích. Komponenta \texttt{FileTransferControls} (viz oddíl \ref{sec:ComponentArchitecture}) poskytuje jednotné rozhraní pro stahování dat jako JSON soubory a jejich načítání pomocí drag-and-drop nebo výběru souboru.

\section{Notifikace a zpětná vazba}
\label{sec:Notifications}

Pro zobrazení upozornění, chybových hlášek a potvrzení úspěšných akcí aplikace využívá knihovnu \emph{react-toastify} \cite{reacttoastify2024}. Tato knihovna poskytuje elegantní toast notifikace, které se objevují v rohu obrazovky a automaticky mizí po nastavené době. Notifikace jsou použity zejména při:
\begin{itemize}
    \item Chybách při parsování vstupů
    \item Úspěšném importu/exportu dat
    \item Varováních při nevalidních operacích
\end{itemize}

Toast notifikace poskytují okamžitou, nenápadnou zpětnou vazbu bez přerušení práce uživatele, což je v souladu s doporučenými praktikami UX designu.

\section{Struktura kódu a modularita}
\label{sec:CodeStructure}

Zdrojový kód aplikace je organizován v adresáři \texttt{src/} následovně:
\begin{itemize}
    \item \textbf{src/} – kořenový adresář obsahující \texttt{App.jsx}, \texttt{main.jsx} a globální styly
    \item \textbf{src/Components/} – všechny React komponenty rozdělené do podadresářů podle účelu:
    \begin{itemize}
        \item \textbf{MCVP/} – modul pro Monotone Circuit Value Problem
        \item \textbf{CombinatorialGame/} – modul pro kombinatorické hry
        \item \textbf{Grammar/} – modul pro bezkontextové gramatiky
        \item \textbf{Conversions/} – komponenty pro vizualizaci převodů mezi problémy
        \item \textbf{Common/} – sdílené komponenty (modální okna, tlačítka, file handling)
        \item \textbf{HPVisual/} – komponenty pro vizualizaci na úvodní stránce
    \end{itemize}
    \item \textbf{src/Hooks/} – custom React hooks pro sdílenou logiku
\end{itemize}

Tato struktura zajišťuje jasné oddělení odpovědností a usnadňuje orientaci v kódu. Každý modul je relativně nezávislý a může být modifikován nebo rozšiřován bez dopadu na ostatní části aplikace.

\section{Vývojové nástroje a kontrola kvality}
\label{sec:DevTools}

Pro zajištění kvality kódu aplikace využívá \emph{ESLint} \cite{eslint2024} – nástroj pro statickou analýzu JavaScriptového kódu. ESLint kontroluje dodržování kódovacích standardů, detekuje potenciální chyby a anti-patterny. Konfigurační soubor \texttt{eslint.config.js} obsahuje pravidla specifická pro React aplikace, včetně doporučení pro používání Hooks a správu závislostí.

Během vývoje je využíván Vite development server s podporou Hot Module Replacement, který umožňuje okamžité promítnutí změn v kódu do běžící aplikace bez nutnosti manuálního obnovení stránky. To výrazně zrychluje vývojový cyklus a umožňuje rychlejší iterace.

\section{Nasazení a distribuce}
\label{sec:Deployment}

Pro produkční nasazení aplikace Vite vytvoří optimalizovaný build pomocí příkazu \texttt{npm run build}. Výsledné soubory v adresáři \texttt{dist/} obsahují minifikovaný JavaScript, CSS a další statické assety. Tyto soubory mohou být nahrány na libovolný webový server nebo hosting platformu podporující statické webové stránky.

Vzhledem k tomu, že aplikace je čistě client-side (veškerá logika běží v prohlížeči uživatele), nevyžaduje serverovou infrastrukturu pro zpracování požadavků. To výrazně zjednodušuje nasazení a hosting, který může být realizován například pomocí služeb GitHub Pages, Netlify, Vercel nebo klasického webového hostingu.
