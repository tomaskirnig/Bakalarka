\chapter{Použité technologie a architektura}
\label{sec:Technologies}

Tato kapitola popisuje technologický základ vytvořené aplikace a její architekturu.

\section{Webové technologie}
\label{sec:WebTechnologies}

Pro nejjednodušší distribuci výukové aplikace byly zvoleny webové technologie, které nabízejí řadu výhod oproti desktopovým nebo mobilním aplikacím. Webová aplikace nevyžaduje instalaci a běží v libovolném moderním webovém prohlížeči, což zajišťuje maximální dostupnost pro uživatele napříč různými platformami a operačními systémy. Dalším přínosem je snadná údržba – aktualizace aplikace se projeví u všech uživatelů bez nutnosti instalace nových verzí koncovým uživatelem.

\section{React a Vite}
\label{sec:ReactVite}

Jako hlavní framework (aplikační rámec) pro vývoj uživatelského rozhraní byl zvolen \emph{React} \cite{react2024}. React je moderní JavaScriptová knihovna vyvinutá společností Meta (dříve Facebook), která umožňuje vytvářet interaktivní uživatelská rozhraní na bázi komponent. Hlavní výhodou Reactu je koncept \emph{reaktivity} – uživatelské rozhraní se automaticky aktualizuje při změně dat bez nutnosti manuální manipulace s DOM (Document Object Model).

React využívá deklarativní přístup k tvorbě UI. Na rozdíl od tradičního imperativního programování, kde vývojář musí krok za krokem instruovat prohlížeč, jak upravit rozhraní, v Reactu stačí popsat, jak má výsledné rozhraní vypadat, a React se postará o potřebné změny v DOM. Tento přístup výrazně zjednodušuje vývoj komplexnějších aplikací a minimalizuje chyby spojené s nekonzistentním stavem.

Pro sestavení aplikace byl použit nástroj (bundler) \emph{Vite} \cite{vite2024}. Vite zajišťuje přípravu všech souborů aplikace (kódu, stylů, obrázků) pro běh v prohlížeči. Během vývoje Vite umožňuje vidět změny v kódu okamžitě po uložení souboru bez nutnosti obnovovat stránku. Pro finální verzi aplikace Vite veškeré soubory optimalizuje a zmenší pro rychlejší načítání.

\section{Architektura a struktura kódu}
\label{sec:ComponentArchitecture}

Aplikace je strukturována jako \emph{Single Page Application} (SPA), kde celá aplikace běží v rámci jedné HTML stránky a navigace mezi jednotlivými moduly probíhá bez opětovného načítání stránky. Hlavní komponenta \texttt{App.jsx} funguje jako kořen aplikační struktury a spravuje globální stav aplikace pomocí React Hooks, především \texttt{useState} pro udržení aktuální stránky a předávaných dat.

Zdrojový kód je rozdělen do tří hlavních modulů podle řešených problémů – MCVP, kombinatorické hry a bezkontextové gramatiky (viz kapitoly \ref{sec:MCVP}, \ref{sec:Games} a \ref{sec:Grammars}). Každý modul obsahuje vlastní logiku a vizualizační komponenty a je organizován do separátních složek podle funckionality:
\begin{itemize}
    \item \textbf{Hlavní komponentu} – kontejnerovou komponentu řídící celý modul
    \item \textbf{Utils} – pomocné funkce obsahující algoritmy (parsery, generátory, evaluátory)
    \item \textbf{InputSelectionComponents} – komponenty pro různé způsoby zadání vstupu
    \item \textbf{Vizualizační komponenty} – komponenty pro grafické zobrazení problémů a jejich řešení
\end{itemize}

Společné prvky jako tlačítka nebo modální okna jsou sdíleny mezi všemi moduly. Pro jendotné nastavení grafů byly vytvořeny pomocné moduly v adresáři \texttt{src/Hooks} – modul \texttt{useGraphColors} spravuje barvy použité v grafických vizualizacích a modul \texttt{useGraphSettings} obshahuje parametry pro vzhled grafů (velikost uzlů, vzdálenosti apod.).

Tato struktura umožňuje snadné úpravy na jednom místě. 

\section{Vizualizace grafů}
\label{sec:GraphVisualization}

Pro vizualizaci grafů používá aplikace knihovnu \emph{react-force-graph} \cite{reactforcegraph2024}, která je postavena na knihovně \emph{D3.js} \cite{d3js2024}. Grafy jsou automaticky rozmístěny pomocí force-directed layoutu – uzly se vzájemně odpuzují a hrany je přitahují k sobě, což vytváří vizuálně přehledné uspořádání. Pro stromové struktury v MCVP modulu (viz kapitola \ref{sec:MCVP}) jsou uzly uspořádány do úrovní shora dolů, s kořenem nahoře a listy dole.

Během vyhodnocování se uzly a hrany v grafu vybarvují podle aktuálního stavu, což uživateli umožňuje sledovat průběh výpočtu.

\section{Stylování a responzivita}
\label{sec:Styling}

Pro vzhled uživatelského rozhraní aplikace využívá framework \emph{Bootstrap 5} \cite{bootstrap2024}. Bootstrap poskytuje předpřipravené styly pro tlačítka, formuláře a další prvky rozhraní. Použití Bootstrapu urychlilo vývoj a zajistilo jednotný vzhled.

\section{Správa vstupů a výstupů}
\label{sec:InputOutput}

Každý modul aplikace nabízí tři způsoby zadání vstupu:
\begin{enumerate}
    \item \textbf{Manuální zadání} – uživatel zadává vstup pomocí textového pole nebo grafického editoru
    \item \textbf{Náhodné generování} – aplikace vygeneruje náhodný příklad podle zadaných parametrů
    \item \textbf{Připravené příklady} – výběr z předpřipravených ukázkových příkladů
\end{enumerate}

Aplikace umožňuje ukládání a načítání dat pomocí souborů ve formátu JSON. Soubory lze načíst přetažením do aplikace nebo výběrem ze složky. Formát nahraného souboru je validován, aby nedošlo k chybám při zpracování neplatných dat.

\section{Notifikace a zpětná vazba}
\label{sec:Notifications}

Pro zobrazení upozornění a chybových hlášek aplikace používá knihovnu \emph{react-toastify} \cite{reacttoastify2024}. Notifikace se objevují v rohu obrazovky a automaticky mizí po chvíli. Aplikace zobrazuje notifikace zejména při chybách ve vstupu, úspěšném uložení dat nebo varováních při neplatných operacích.

Notifikace poskytují okamžitou zpětnou vazbu pro uživatele.

\begin{figure}[htbp]
    \centering
    \includegraphics[width=0.5\textwidth]{IMGs/ToastifyMessage.png}
    \caption{Příklad notifikační zprávy v aplikaci}
    \label{fig:toastify-message}
\end{figure}

\section{Vývojové nástroje}
\label{sec:DevTools}

Pro zajištění kvality kódu aplikace používá nástroj \emph{ESLint} \cite{eslint2024}, který kontroluje dodržování standardů a detekuje potenciální chyby v kódu.

\section{Nasazení aplikace}
\label{sec:Deployment}

Pro zveřejnění aplikace nástroj Vite připraví a optimalizuje všechny soubory. Protože veškerá logika aplikace běží přímo v prohlížeči uživatele, není potřeba žádný speciální server.

Aplikace je nasazena pomocí služby \emph{Vercel}, která je přímo propojena s GitHub repozitářem. Při každé změně v hlavní větvi repozitáře dojde automaticky k novému nasazení aplikace. Toto řešení výrazně zjednodušuje proces aktualizace a zajišťuje, že živá verze aplikace je vždy aktuální.
