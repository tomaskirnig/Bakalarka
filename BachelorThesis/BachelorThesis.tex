% Nejprve uvedeme tridu dokumentu s volbami
\documentclass[czech,bachelor]{diploma}
% Dalsi doplnujici baliky maker
\usepackage[autostyle=true,czech=quotes]{csquotes} % korektni sazba uvozovek, podpora pro balik biblatex
\usepackage[backend=biber, style=iso-numeric, alldates=iso]{biblatex} % bibliografie
\usepackage{dcolumn} % sloupce tabulky s ciselnymi hodnotami
\usepackage{subfig} % makra pro "podobrazky" a "podtabulky"
\usepackage[cpp]{diplomalst}

% Zadame pozadovane vstupy pro generovani titulnich stran.
\ThesisAuthor{Tomáš Kirnig}
% Použité knihovny
%react-force-graph

\ThesisSupervisor{Ing. Martin Kot, Ph.D.}

\CzechThesisTitle{Komponenta výukového serveru TI - P-úplné problémy}

\EnglishThesisTitle{Component of Teaching Server for Theoretical Computer Science - Pcomplete problems
}

\SubmissionYear{2025}

\ThesisAssignmentFileName{ZadáníPráce.pdf}

% Pokud nechceme nikomu dekovat makro zapoznamkujeme.
% \Acknowledgement{Rád bych na tomto místě poděkoval všem, kteří mi s prací pomohli, protože bez nich by tato práce nevznikla.}

\CzechAbstract{Tato bakalářská práce se zabývá vývojem výukové webové aplikace pro demonstraci P-úplných problémů. Hlavním cílem je usnadnit studentům pochopení a procvičení takových úloh, mezi které patří například \emph{Monotone Circuit Value Problem} (MCVP) a dvě další vybrané P-úplné úlohy. V rámci práce jsou implementovány moduly pro interaktivní zadávání vstupů uživatelem (ručně, náhodnou generací nebo výběrem z připravené sady) a simulaci výpočtu řešení. Uživatel má rovněž možnost sledovat krokový převod instance MCVP na jiné P-úplné problémy a následně i jejich samotné řešení. Výsledná aplikace demonstruje principy teoretické informatiky včetně pojmu P-úplnosti a poskytuje rozšiřitelný základ pro výukové účely.}

\CzechKeywords{Teoretická informatika, P-úplné problémy, Monotone Circuit Value Problem, webová aplikace, simulace, převod instancí}

\EnglishAbstract{This bachelor’s thesis focuses on the development of a teaching-oriented web application for illustrating P-complete problems. The main goal is to facilitate students’ understanding and practice of such tasks, including the \emph{Monotone Circuit Value Problem} (MCVP) and two additional P-complete problems. The project implements modules for interactive input of problem instances (manually, via random generation, or by selecting from a pre-defined set) and provides a simulation of their solutions. Users can also observe a step-by-step reduction from an MCVP instance to other P-complete problems and subsequently explore how those are solved. The resulting application demonstrates key concepts of theoretical computer science, including the notion of P-completeness, and provides a flexible basis for educational use.}

\EnglishKeywords{Theoretical computer science, P-complete problems, Monotone Circuit Value Problem, web application, simulation, instance reduction}

\AddAcronym{MCVP}{Monotone Circuit Value Problem}
\AddAcronym{PC}{P-complete}
\AddAcronym{BFS}{Breadth-First Search}
\AddAcronym{DFS}{Depth-First Search}

\addbibresource{bibliography.bib}
\addbibresource{coffee.bib}

% Novy druh tabulkoveho sloupce, ve kterem jsou cisla zarovnana podle desetinne carky
\newcolumntype{d}[1]{D{,}{,}{#1}}


% Zacatek dokumentu
\begin{document}

% Nechame vysazet titulni strany.
\MakeTitlePages

% Jsou v praci obrazky? Pokud ano vysazime jejich seznam a odstrankujeme.
% Pokud ne smazeme nasledujici dve makra.
\listoffigures
\clearpage

% Jsou v praci tabulky? Pokud ano vysazime jejich seznam a odstrankujeme.
% Pokud ne smazeme nasledujici dve makra.
\listoftables
\clearpage

% A nasleduje text zaverecne prace.
\chapter{Úvod}
\label{sec:Introduction}

Teoretická informatika poskytuje způsob jak zkoumat, porozumět a optimalizovat možnosti a limity výpočetních systémů. Hlavním bodem tohoto zkoumání je teorie složitosti, která klasifikuje problémy na základě zdrojů potřebných pro jejich vyřešení, jako je čas a paměť. Jednou z nejvýznamnějších složitostních tříd je třída \( P \), zahrnující problémy řešitelné v polynomiálním čase – tedy takové, jejichž časová složitost lze vyjádřit jako \( O(n^k) \) pro nějakou konstantu \( k \) (například \( k = 2 \) pro kvadratickou složitost, \( k = 3 \) pro kubickou), kde \( n \) je velikost vstupu – na deterministickém Turingově stroji \cite{papadimitriou1993}. Ačkoliv jsou problémy v této třídě tradičně považovány za „efektivně řešitelné", existují mezi nimi zásadní rozdíly, hlavně jestli uvažujeme o možnostech jejich paralelizace.

V tomto kontextu hraje klíčovou roli koncept \emph{P-úplnosti} (P-completeness). P-úplné problémy představují výpočetně nejnáročnější úlohy v rámci třídy \( P \). Jsou to problémy, na které lze v logaritmickém prostoru redukovat jakýkoliv jiný problém z \( P \). Důležitost této klasifikace spočívá v hypotéze, že P-úplné problémy pravděpodobně nelze efektivně paralelizovat (nepatří do třídy \( NC \)), a jejich řešení je tedy inherentně sekvenční \cite{arora2009}. Studium těchto problémů tak umožňuje lépe vymezit hranici mezi úlohami, které lze urychlit paralelním zpracováním, a těmi, které vyžadují striktně sériový postup.

Výuka těchto abstraktních konceptů však představuje značnou výzvu. Porozumění formálním definicím redukcí a složitostních tříd pouze prostřednictvím statického textu může být pro studenty obtížné. Zatímco pro základní algoritmy existuje řada vizualizačních nástrojů, oblast složitostní teorie – a konkrétně P-úplnost – zůstává v edukačním softwaru často opomíjena. Cílem této práce je tuto mezeru zaplnit návrhem a implementací interaktivní výukové komponenty, která umožní dynamickou demonstraci těchto principů.

Jako referenční problém byl zvolen \emph{Monotone Circuit Value Problem} (MCVP), který spočívá ve vyhodnocení logického obvodu složeného z monotonních hradel a přirozeně modeluje tok výpočtu \cite{miyano1990}. Pro demonstraci univerzálnosti konceptu P-úplnosti aplikace implementuje také simulace dvou dalších problémů z odlišných domén:
\begin{itemize}
    \item \textbf{Kombinatorické hry na grafu:} Úloha analyzující existenci vítězné strategie v deterministických hrách dvou hráčů.
    \item \textbf{Prázdnost bezkontextových gramatik:} Problém rozhodující, zda daná gramatika generuje neprázdný jazyk \cite{sawa}.
\end{itemize}

Hlavním přínosem vytvořené aplikace je možnost vizualizace nejen samotného řešení těchto úloh, ale především \emph{převodů} (redukcí) mezi nimi. Uživatel může sledovat krokovou transformaci instance MCVP na instanci hry nebo gramatiky, což názorně ilustruje princip polynomiálních redukcí a vzájemnou převoditelnost P-úplných problémů \cite{arora2009}.

Výsledkem práce je moderní webová aplikace navržená jako flexibilní nástroj pro výuku. Umožňuje uživatelům pracovat s vlastními vstupy, generovat náhodné instance pro testování a využívat předpřipravené sady úloh pro řízené studium.

Text práce je členěn do logických celků. Po úvodu následuje kapitola věnovaná použitým technologiím a architektuře aplikace. Jádro práce tvoří tři kapitoly, z nichž každá se detailně věnuje jednomu z implementovaných problémů: nejprve Monotone Circuit Value Problem (MCVP), následně kombinatorické hry na grafu a nakonec problém prázdnosti bezkontextových gramatik. Závěr práce shrnuje dosažené výsledky a navrhuje možnosti dalšího rozšíření.

\endinput
% \input{Chapters/SampleChapter1.tex}
% \input{Chapters/SampleChapter2.tex}
\input{Chapters/TechnicalDetails.tex}
% \chapter{Závěr}
\label{sec:Conclusion}

Cílem této práce bylo vytvořit interaktivní webovou aplikaci pro vizualizaci a demonstraci převoditelnosti a řešení P-úplných problémů na příkladu tří vybraných P-úplných problémů: Monotone Circuit Value Problem, kombinatorických her na grafu a problému neprázdnosti jazyka bezkontextové gramatiky.

\section{Dosažené výsledky}
\label{sec:ConclusionAchievements}

Výsledná aplikace splňuje všechny stanovené cíle a poskytuje jednotné prostředí pro práci se třemi P-úplnými problémy. Pro každý problém je implementováno kompletní řešení zahrnující:

\begin{itemize}
    \item \textbf{Flexibilní vstupní systém:} Uživatelé mohou zadávat instance problémů třemi způsoby – manuálním zadáním, generováním náhodných instancí nebo pomocí předpřipravených příkladů. Tyto možnosti umožňují jak experimentování s vlastními příklady, tak rychlé testování algoritmu na náhodných datech.
    
    \item \textbf{Vizualizaci řešení:} Každý problém je doprovázen grafickou reprezentací, která využívá knihovnu \emph{react-force-graph-2d} pro interaktivní zobrazení grafových struktur. Uživatel může manipulovat se zobrazenými grafy, přibližovat je a přesouvat uzly pro lepší orientaci.
    
    \item \textbf{Krokové vyhodnocení:} Implementace krokovatelného průchodu algoritmů umožňuje sledovat každý jednotlivý krok výpočtu s textovým vysvětlením.
    
    \item \textbf{Export a import dat:} Možnost ukládání a načítání instancí problémů ve formátu JSON podporuje sdílení příkladů a vytváření knihoven testovacích případů.
\end{itemize}

Aplikace taktéž demonstruje dva konkrétní převody z MCVP na kombinatorickou hru a na bezkontextovou gramatiku. Oba převody jsou implementovány krokovatelně, takže uživatel může sledovat, jak se jednotlivé uzly obvodu transformují na odpovídající struktury v cílovém problému. Tato vizualizace názorně ilustruje techniku polynomiálních redukcí a vzájemnou převoditelnost P-úplných problémů.

\section{Vzdělávací přínos}
\label{sec:ConclusionEducational}

Aplikace představuje vzdělávací nástroj pro výuku teorie složitosti s interaktivním přístupem. Krokové vyhodnocení s textovými vysvětleními umožňuje pochopit fungování algoritmů na konkrétních příkladech, zatímco vizualizace převodů demonstruje vzájemnou převoditelnost P-úplných problémů. Generování náhodných instancí pak podporuje experimentální učení a pozorování chování algoritmů na různých strukturách vstupů.

\section{Možnosti dalšího rozvoje}
\label{sec:ConclusionFutureWork}

Přestože aplikace poskytuje funkční implementaci všech plánovaných funkcí, existuje prostor pro další rozšíření:

\begin{itemize}
    \item \textbf{Další P-úplné problémy:} Aplikace by mohla být rozšířena o další P-úplné problémy, jako je například vyhodnocování booleovských formulí v konjunktivní normální formě nebo problém dosažitelnosti v grafech s omezenou šířkou.
    
    \item \textbf{Více převodů:} Implementace dalších směrů převodů – například z kombinatorických her na gramatiky nebo opačným směrem z gramatik na MCVP – by poskytla kompletnější obraz vzájemné převoditelnosti těchto problémů.
    
    \item \textbf{Výkonnostní optimalizace:} Pro velmi velké instance (stovky uzlů) by mohly být implementovány optimalizace vykreslování a výpočtu, například pomocí virtualizace zobrazení nebo progresivního vykreslování.
\end{itemize}

Výsledná aplikace může sloužit jako doplněk k tradičním výukovým materiálům v kurzech teorie složitosti a teoretické informatiky, kde pomáhá studentům lépe pochopit abstraktní koncepty prostřednictvím konkrétních interaktivních příkladů.

\endinput


% Seznam literatury
\printbibliography[title={Literatura}, heading=bibintoc]

% Prilohy
\appendix
% \input{Chapters/Appendix1.tex}
% \input{Chapters/Appendix2.tex}

% Priloha vlozena primo do hlavniho LaTeX souboru. Ne vsechny prilohy je nutne mit ve zvlastnich souborech.
\chapter{Dlouhý zdrojový kód}
% \lstinputlisting[label=src:CppExternal,caption={Dlouhý zdrojový kód v jazyce C++ načtený s externího souboru}]{SourceCodes/ArraySortingAlgorithms.cpp}

\end{document}
