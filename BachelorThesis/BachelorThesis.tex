% Nejprve uvedeme tridu dokumentu s volbami
\documentclass[czech,bachelor]{diploma}
% Dalsi doplnujici baliky maker
\usepackage[autostyle=true,czech=quotes]{csquotes} % korektni sazba uvozovek, podpora pro balik biblatex
\usepackage[backend=biber, style=iso-numeric, alldates=iso, sorting=nty]{biblatex} % bibliografie
\usepackage{dcolumn} % sloupce tabulky s ciselnymi hodnotami
\usepackage{subfig} % makra pro "podobrazky" a "podtabulky"
\usepackage[cpp]{diplomalst}
\usepackage[utf8]{inputenc} % Set input encoding to UTF-8
\usepackage[T1]{fontenc} % Use T1 font encoding for better support of special characters

% Zadame pozadovane vstupy pro generovani titulnich stran.
\ThesisAuthor{Tomáš Kirnig}
% Použité knihovny
%react-force-graph

\ThesisSupervisor{Ing. Martin Kot, Ph.D.}

\CzechThesisTitle{Komponenta výukového serveru TI - P-úplné problémy}

\EnglishThesisTitle{Component of Teaching Server for Theoretical Computer Science - Pcomplete problems
}

\SubmissionYear{2025}

\ThesisAssignmentFileName{ZadáníPráce.pdf}

% Pokud nechceme nikomu dekovat makro zapoznamkujeme.
% \Acknowledgement{Rád bych na tomto místě poděkoval všem, kteří mi s prací pomohli, protože bez nich by tato práce nevznikla.}

\CzechAbstract{Tato bakalářská práce se zabývá vývojem výukové webové aplikace pro demonstraci P-úplných problémů. Cílem je usnadnit studentům pochopení a procvičování těchto problémů. Aplikace se zaměřuje na tři P-úplné problémy: \emph{Monotone Circuit Value Problem} (MCVP), \emph{problém neprázdnosti bezkontextové gramatiky} a \emph{kombinatorické hry na grafech}. Práce zahrnuje implementaci modulů pro interaktivní zadávání vstupů (ručně, náhodnou generací nebo výběrem z připravené sady) a simulaci jejich řešení. Uživatel může také sledovat krokový převod instance MCVP na další zmíněné problémy a následně jejich řešení.}

\CzechKeywords{Teoretická informatika, P-úplné problémy, Monotone Circuit Value Problem, webová aplikace, simulace, převod instancí}

\EnglishAbstract{This bachelor's thesis focuses on the development of a teaching-oriented web application for illustrating P-complete problems. The main goal is to facilitate students' understanding and practice of such tasks. The application focuses on the \emph{Monotone Circuit Value Problem} (MCVP), \emph{Empty Grammar}, and the \emph{Combinatorial Game}. The project implements modules for interactive input of problem instances (manually, via random generation, or by selecting from a pre-defined set) and provides a simulation of their solutions. Users can also observe a step-by-step reduction from an MCVP instance to the other mentioned problems and subsequently explore how those are solved. The resulting application demonstrates key concepts of theoretical computer science, including the notion of P-completeness, and provides a flexible basis for educational use.}

\EnglishKeywords{Theoretical computer science, P-complete problems, Monotone Circuit Value Problem, web application, simulation, instance reduction}

\AddAcronym{MCVP}{Monotone Circuit Value Problem}
\AddAcronym{P}{Třída problémů řešitelných v polynomiálním čase}
\AddAcronym{NC}{Nick's Class -- třída efektivně paralelizovatelných problémů}
\AddAcronym{BFS}{Breadth First Search -- Algoritmus průchodu do šířky}
\AddAcronym{CFG}{Context-Free Grammar -- Bezkontextová gramatika}
\AddAcronym{DAG}{Directed Acyclic Graph -- Orientovaný acyklický graf}
\AddAcronym{DFS}{Depth First Search -- Algoritmus průchodu do hloubky}
\AddAcronym{DOM}{Document Object Model -- Objektový model dokumentu}
\AddAcronym{HTML}{Hypertext Markup Language -- Hypertextový značkovací jazyk}
\AddAcronym{ID}{Identifier -- Identifikátor}
\AddAcronym{JSON}{JavaScript Object Notation -- Objektový zápis JavaScriptu}
\AddAcronym{LS}{Levá strana}
\AddAcronym{PS}{Pravá strana}
\AddAcronym{SPA}{Single Page Application -- Jednostránková webová aplikace}
\AddAcronym{TI}{Teoretická informatika}
\AddAcronym{UI}{User Interface -- Uživatelské rozhraní}

\addbibresource{bibliography.bib}

% Novy druh tabulkoveho sloupce, ve kterem jsou cisla zarovnana podle desetinne carky
\newcolumntype{d}[1]{D{,}{,}{#1}}


% Zacatek dokumentu
\begin{document}

% Nechame vysazet titulni strany.
\MakeTitlePages

% Jsou v praci obrazky? Pokud ano vysazime jejich seznam a odstrankujeme.
% Pokud ne smazeme nasledujici dve makra.
\listoffigures
\clearpage

% Jsou v praci tabulky? Pokud ano vysazime jejich seznam a odstrankujeme.
% Pokud ne smazeme nasledujici dve makra.
% \listoftables
% \clearpage

% A nasleduje text zaverecne prace.
\chapter{Úvod}
\label{sec:Introduction}

Teoretická informatika poskytuje způsob jak zkoumat, porozumět a optimalizovat možnosti a limity výpočetních systémů. Hlavním bodem tohoto zkoumání je teorie složitosti, která klasifikuje problémy na základě zdrojů potřebných pro jejich vyřešení, jako je čas a paměť. Jednou z nejvýznamnějších složitostních tříd je třída \( P \), zahrnující problémy řešitelné v polynomiálním čase – tedy takové, jejichž časová složitost lze vyjádřit jako \( O(n^k) \) pro nějakou konstantu \( k \) (například \( k = 2 \) pro kvadratickou složitost, \( k = 3 \) pro kubickou), kde \( n \) je velikost vstupu – na deterministickém Turingově stroji \cite{papadimitriou1993}. Ačkoliv jsou problémy v této třídě tradičně považovány za „efektivně řešitelné", existují mezi nimi zásadní rozdíly, hlavně jestli uvažujeme o možnostech jejich paralelizace.

V tomto kontextu hraje klíčovou roli koncept \emph{P-úplnosti} (P-completeness). P-úplné problémy představují výpočetně nejnáročnější úlohy v rámci třídy \( P \). Jsou to problémy, na které lze v logaritmickém prostoru redukovat jakýkoliv jiný problém z \( P \). Důležitost této klasifikace spočívá v hypotéze, že P-úplné problémy pravděpodobně nelze efektivně paralelizovat (nepatří do třídy \( NC \)), a jejich řešení je tedy inherentně sekvenční \cite{arora2009}. Studium těchto problémů tak umožňuje lépe vymezit hranici mezi úlohami, které lze urychlit paralelním zpracováním, a těmi, které vyžadují striktně sériový postup.

Výuka těchto abstraktních konceptů však představuje značnou výzvu. Porozumění formálním definicím redukcí a složitostních tříd pouze prostřednictvím statického textu může být pro studenty obtížné. Zatímco pro základní algoritmy existuje řada vizualizačních nástrojů, oblast složitostní teorie – a konkrétně P-úplnost – zůstává v edukačním softwaru často opomíjena. Cílem této práce je tuto mezeru zaplnit návrhem a implementací interaktivní výukové komponenty, která umožní dynamickou demonstraci těchto principů.

Jako referenční problém byl zvolen \emph{Monotone Circuit Value Problem} (MCVP), který spočívá ve vyhodnocení logického obvodu složeného z monotonních hradel a přirozeně modeluje tok výpočtu \cite{miyano1990}. Pro demonstraci univerzálnosti konceptu P-úplnosti aplikace implementuje také simulace dvou dalších problémů z odlišných domén:
\begin{itemize}
    \item \textbf{Kombinatorické hry na grafu:} Úloha analyzující existenci vítězné strategie v deterministických hrách dvou hráčů.
    \item \textbf{Prázdnost bezkontextových gramatik:} Problém rozhodující, zda daná gramatika generuje neprázdný jazyk \cite{sawa}.
\end{itemize}

Hlavním přínosem vytvořené aplikace je možnost vizualizace nejen samotného řešení těchto úloh, ale především \emph{převodů} (redukcí) mezi nimi. Uživatel může sledovat krokovou transformaci instance MCVP na instanci hry nebo gramatiky, což názorně ilustruje princip polynomiálních redukcí a vzájemnou převoditelnost P-úplných problémů \cite{arora2009}.

Výsledkem práce je moderní webová aplikace navržená jako flexibilní nástroj pro výuku. Umožňuje uživatelům pracovat s vlastními vstupy, generovat náhodné instance pro testování a využívat předpřipravené sady úloh pro řízené studium.

Text práce je členěn do logických celků. Po úvodu následuje kapitola věnovaná použitým technologiím a architektuře aplikace. Jádro práce tvoří tři kapitoly, z nichž každá se detailně věnuje jednomu z implementovaných problémů: nejprve Monotone Circuit Value Problem (MCVP), následně kombinatorické hry na grafu a nakonec problém prázdnosti bezkontextových gramatik. Závěr práce shrnuje dosažené výsledky a navrhuje možnosti dalšího rozšíření.

\endinput

\chapter{Použité technologie a architektura}
\label{sec:Technologies}

Tato kapitola popisuje technologický základ vytvořené aplikace a její architekturu. Výběr technologií byl veden požadavky na moderní, interaktivní a snadno rozšiřitelné řešení vhodné pro výukové účely.

\section{Webové technologie}
\label{sec:WebTechnologies}

Pro implementaci výukové aplikace jsme zvolili webové technologie, které nabízejí řadu výhod oproti desktopovým nebo mobilním aplikacím. Webová aplikace nevyžaduje instalaci a běží v libovolném moderním webovém prohlížeči, což zajišťuje maximální dostupnost pro uživatele napříč různými platformami a operačními systémy. Dalším přínosem je snadná údržba – aktualizace aplikace se projeví okamžitě u všech uživatelů bez nutnosti distribuce nových verzí.

\section{React a Vite}
\label{sec:ReactVite}

Jako hlavní framework pro vývoj uživatelského rozhraní byl zvolen \emph{React} \cite{react2024}. React je moderní JavaScriptová knihovna vyvinutá společností Meta (dříve Facebook), která umožňuje vytvářet interaktivní uživatelská rozhraní na bázi komponent. Hlavní výhodou Reactu je koncept \emph{reaktivity} – uživatelské rozhraní se automaticky aktualizuje při změně dat bez nutnosti manuální manipulace s DOM (Document Object Model).

React využívá deklarativní přístup k tvorbě UI. Na rozdíl od tradičního imperativního programování, kde vývojář musí krok za krokem instruovat prohlížeč, jak upravit rozhraní, v Reactu stačí popsat, jak má výsledné rozhraní vypadat, a React se postará o potřebné změny v DOM. Tento přístup výrazně zjednodušuje vývoj komplexnějších aplikací a minimalizuje chyby spojené s nekonzistentním stavem.

Pro sestavení a vývoj aplikace jsme použili \emph{Vite} \cite{vite2024}. Vite je moderní \emph{bundler} – tedy nástroj, který spojuje všechny zdrojové soubory aplikace (JavaScript, CSS, obrázky) do optimalizované podoby připravené pro odeslání do prohlížeče. Během vývoje Vite nabízí významnou výhodu: místo zdlouhavého spojování všech souborů do jednoho velkého balíku servíruje jednotlivé moduly přímo do prohlížeče v jejich původní podobě. To výrazně urychluje spouštění vývojového serveru a umožňuje okamžité promítnutí změn v kódu (tzv. hot module replacement). Pro produkční nasazení pak Vite vytvoří klasický optimalizovaný build, kde jsou všechny soubory sloučeny, zmenšeny a připraveny pro rychlé načítání.

\section{Správa stavu a komponentová architektura}
\label{sec:ComponentArchitecture}

Aplikace je strukturována jako \emph{Single Page Application} (SPA), kde celá aplikace běží v rámci jedné HTML stránky a navigace mezi jednotlivými moduly probíhá bez opětovného načítání stránky. Hlavní komponenta \texttt{App.jsx} funguje jako kořen aplikační struktury a spravuje globální stav aplikace pomocí React Hooks, především \texttt{useState} pro udržení aktuální stránky a předávaných dat.

Architektura je založena na principu \emph{unidirectional data flow} (jednosměrný tok dat), kdy data tečou z rodičovské komponenty do potomků prostřednictvím properties (props), zatímco akce a události se propagují zpět nahoru pomocí callback funkcí. Tento přístup zajišťuje předvídatelné chování aplikace a usnadňuje ladění.

Aplikace je rozdělena do logických modulů podle jednotlivých problémů (viz kapitoly \ref{sec:MCVP}, \ref{sec:Games} a \ref{sec:Grammars}). Každý modul je organizován do vlastní složky obsahující:
\begin{itemize}
    \item \textbf{Hlavní komponentu} – kontejnerovou komponentu řídící celý modul
    \item \textbf{Utils} – pomocné funkce obsahující algoritmy (parsery, generátory, evaluátory)
    \item \textbf{InputSelectionComponents} – komponenty pro různé způsoby zadání vstupu
    \item \textbf{Vizualizační komponenty} – komponenty pro grafické zobrazení problémů a jejich řešení
\end{itemize}

Pro sdílení společné funkcionality mezi moduly byly vytvořeny \emph{custom hooks} (vlastní React Hooks) v adresáři \texttt{src/Hooks}. Hook \texttt{useGraphColors} centralizuje správu barev používaných ve vizualizacích a načítá je z CSS proměnných, což umožňuje snadnou změnu barevného schématu. Hook \texttt{useGraphSettings} poskytuje konfigurační parametry pro force-directed grafy, jako jsou poloměry uzlů, síly odpudivosti a vzdálenosti hran.

\section{Vizualizace grafů}
\label{sec:GraphVisualization}

Pro vizualizaci stromových a grafových struktur používá aplikace knihovnu \emph{react-force-graph-2d} \cite{reactforcegraph2024}, která je React wrapperem nad výkonnou knihovnou \emph{D3.js} (Data-Driven Documents) \cite{d3js2024}. D3.js je de facto standardem pro tvorbu datových vizualizací ve webovém prostředí a poskytuje rozsáhlé možnosti pro manipulaci s DOM na základě dat.

\subsection{Force-directed layout}
\label{sec:ForceDirectedLayout}

Klíčovým konceptem využívaným pro rozmístění uzlů v grafu je \emph{force-directed layout} (rozložení řízené silami). Tento algoritmus modeluje graf jako fyzikální systém, kde uzly představují nabitá tělesa odpudící se navzájem a hrany fungují jako pružiny přitahující spojené uzly k sobě. Simulace běží iterativně, dokud systém nedosáhne stavu s minimální energií, tedy vizuálně příjemného a čitelného rozložení.

Konkrétně aplikace využívá následující síly:
\begin{itemize}
    \item \textbf{Link force} – síla pružin mezi spojenými uzly, udržuje požadovanou vzdálenost hran
    \item \textbf{Charge force} – odpudivá síla mezi všemi uzly, zabraňuje jejich překrývání
    \item \textbf{Collision force} – detekuje a řeší kolize mezi uzly na základě jejich poloměrů
    \item \textbf{Center force} – udržuje celý graf vycentrovaný v zobrazovací ploše
\end{itemize}

Pro stromové struktury v MCVP modulu (viz kapitola \ref{sec:MCVP}) je aktivován speciální \emph{DAG mód} (Directed Acyclic Graph), který automaticky organizuje uzly do hierarchických úrovní shora dolů (\texttt{dagMode="td"}). To zajišťuje, že výraz je vizualizován tak, jak je běžné v teoretické informatice – kořen nahoře, listy dole.

\subsection{Vlastní vykreslování}
\label{sec:CustomRendering}

Knihovna react-force-graph-2d umožňuje definovat vlastní vykreslovací funkce (\texttt{nodeCanvasObject}, \texttt{linkCanvasObjectMode}), což aplikace využívá pro vizualizaci stavů uzlů a hran během evaluace. Například v MCVP modulu se uzly vybarvují podle jejich typu (proměnné vs. operátory) a vyhodnoceného stavu (0 nebo 1), zatímco hrany mění barvu podle toho, zda již byly zpracovány v průchodu stromem.

Vykreslování probíhá na HTML5 Canvas elementu, který nabízí vysoký výkon i pro grafy s desítkami či stovkami uzlů. Oproti SVG přístupu je Canvas renderování efektivnější pro časté překreslování, což je klíčové pro animace krokování algoritmy.

\section{Stylování a responzivita}
\label{sec:Styling}

Pro stylování uživatelského rozhraní aplikace využívá framework \emph{Bootstrap 5} \cite{bootstrap2024}. Bootstrap poskytuje předpřipravené CSS třídy pro tvorbu responzivního layoutu, komponent jako jsou tlačítka, formuláře, modální okna a navigační prvky. Použití Bootstrapu výrazně urychlilo vývoj a zajistilo konzistentní vzhled napříč celou aplikací.

Responzivita je implementována pomocí Bootstrapového grid systému založeného na Flexboxu. Layout se automaticky přizpůsobuje velikosti obrazovky, přičemž na mobilních zařízeních se navigace transformuje do off-canvas menu, které se vysouvá ze strany obrazovky. Toto řešení zajišťuje dobrou použitelnost aplikace i na tabletech a smartphonech.

Vedle Bootstrapu jsou specifické styly definovány v centrálním souboru \texttt{style.css}, který využívá CSS custom properties (proměnné) pro správu barevného schématu. Tento přístup umožňuje snadnou změnu celého vzhledu aplikace změnou několika proměnných.

\section{Správa vstupů a výstupů}
\label{sec:InputOutput}

Každý modul aplikace implementuje tři základní způsoby zadání vstupu:
\begin{enumerate}
    \item \textbf{Manuální zadání} – uživatel zadává vstup prostřednictvím textového pole (např. výraz v notaci pro MCVP) nebo interaktivního grafického editoru
    \item \textbf{Náhodné generování} – algoritmus generuje náhodnou instanci problému na základě parametrů zadaných uživatelem
    \item \textbf{Připravené sady} – výběr z předpřipravených příkladů uložených ve formátu JSON v adresáři \texttt{Sady/}
\end{enumerate}

Pro import a export dat aplikace používá formát JSON, který je standardním formátem pro výměnu dat ve webových aplikacích. Komponenta \texttt{FileTransferControls} (viz oddíl \ref{sec:ComponentArchitecture}) poskytuje jednotné rozhraní pro stahování dat jako JSON soubory a jejich načítání pomocí drag-and-drop nebo výběru souboru.

\section{Notifikace a zpětná vazba}
\label{sec:Notifications}

Pro zobrazení upozornění, chybových hlášek a potvrzení úspěšných akcí aplikace využívá knihovnu \emph{react-toastify} \cite{reacttoastify2024}. Tato knihovna poskytuje elegantní toast notifikace, které se objevují v rohu obrazovky a automaticky mizí po nastavené době. Notifikace jsou použity zejména při:
\begin{itemize}
    \item Chybách při parsování vstupů
    \item Úspěšném importu/exportu dat
    \item Varováních při nevalidních operacích
\end{itemize}

Toast notifikace poskytují okamžitou, nenápadnou zpětnou vazbu bez přerušení práce uživatele, což je v souladu s doporučenými praktikami UX designu.

\section{Struktura kódu a modularita}
\label{sec:CodeStructure}

Zdrojový kód aplikace je organizován v adresáři \texttt{src/} následovně:
\begin{itemize}
    \item \textbf{src/} – kořenový adresář obsahující \texttt{App.jsx}, \texttt{main.jsx} a globální styly
    \item \textbf{src/Components/} – všechny React komponenty rozdělené do podadresářů podle účelu:
    \begin{itemize}
        \item \textbf{MCVP/} – modul pro Monotone Circuit Value Problem
        \item \textbf{CombinatorialGame/} – modul pro kombinatorické hry
        \item \textbf{Grammar/} – modul pro bezkontextové gramatiky
        \item \textbf{Conversions/} – komponenty pro vizualizaci převodů mezi problémy
        \item \textbf{Common/} – sdílené komponenty (modální okna, tlačítka, file handling)
        \item \textbf{HPVisual/} – komponenty pro vizualizaci na úvodní stránce
    \end{itemize}
    \item \textbf{src/Hooks/} – custom React hooks pro sdílenou logiku
\end{itemize}

Tato struktura zajišťuje jasné oddělení odpovědností a usnadňuje orientaci v kódu. Každý modul je relativně nezávislý a může být modifikován nebo rozšiřován bez dopadu na ostatní části aplikace.

\section{Vývojové nástroje a kontrola kvality}
\label{sec:DevTools}

Pro zajištění kvality kódu aplikace využívá \emph{ESLint} \cite{eslint2024} – nástroj pro statickou analýzu JavaScriptového kódu. ESLint kontroluje dodržování kódovacích standardů, detekuje potenciální chyby a anti-patterny. Konfigurační soubor \texttt{eslint.config.js} obsahuje pravidla specifická pro React aplikace, včetně doporučení pro používání Hooks a správu závislostí.

Během vývoje je využíván Vite development server s podporou Hot Module Replacement, který umožňuje okamžité promítnutí změn v kódu do běžící aplikace bez nutnosti manuálního obnovení stránky. To výrazně zrychluje vývojový cyklus a umožňuje rychlejší iterace.

\section{Nasazení a distribuce}
\label{sec:Deployment}

Pro produkční nasazení aplikace Vite vytvoří optimalizovaný build pomocí příkazu \texttt{npm run build}. Výsledné soubory v adresáři \texttt{dist/} obsahují minifikovaný JavaScript, CSS a další statické assety. Tyto soubory mohou být nahrány na libovolný webový server nebo hosting platformu podporující statické webové stránky.

Vzhledem k tomu, že aplikace je čistě client-side (veškerá logika běží v prohlížeči uživatele), nevyžaduje serverovou infrastrukturu pro zpracování požadavků. To výrazně zjednodušuje nasazení a hosting, který může být realizován například pomocí služeb GitHub Pages, Netlify, Vercel nebo klasického webového hostingu.

\chapter{Monotone Circuit Value Problem}
\label{sec:MCVP}

% Zde bude teorie k MCVP, popis implementace a redukcí.

\chapter{Kombinatorická hra}
\label{sec:Games}

\section{Teoretický základ}
\label{sec:GamesTheory}

Kombinatorická hra dvou hráčů na orientovaném grafu je další příklad P-úplného problému \cite{sawa}. Hra má tyto vlastnosti:

\begin{itemize}
    \item \textbf{Dva hráči:} Každé pole grafu má specifikováno, který hráč je na tahu – Hráč I (první hráč) nebo Hráč II (druhý hráč).
    \item \textbf{Konečná pozice:} Hra končí, když je hráč na tahu v pozici bez možných dalších tahů.
\end{itemize}

Problém spočívá v rozhodnutí, zda Hráč I má výherní strategii ze zadané počáteční pozice:

\begin{itemize}
    \item \textbf{Vstup:} Orientovaný graf (DG), kde každý uzel reprezentuje herní pozici přiřazenou některému z hráčů, hrany reprezentují možné tahy a jeden uzel je označen jako počáteční pozice.
    \item \textbf{Výstup:} Rozhodnutí, zda Hráč I má výherní strategii ze startovní pozice.
\end{itemize}

\textbf{Pravidla hry:} Hra končí, když se dostane do pozice, kde hráč na tahu nemá žádný možný tah – tento hráč pak prohrává. Hráči se střídají v tazích podle toho, jaký hráč je přiřazen aktuální pozici: je-li uzel přiřazen Hráči I, táhne Hráč I; je-li přiřazen Hráči II, táhne Hráč II.

\textbf{Výherní strategie:} Výherní strategie pro Hráče I je taková posloupnost tahů, která garantuje výhru Hráče I bez ohledu na to, jak hraje Hráč II. Jinými slovy, Hráč I má výherní strategii, pokud existuje způsob, jak vždy volit tahy tak, aby se hra dostala do pozice, kde je Hráč II na tahu a nemá žádný možný tah \cite{sawa}.

Hráč I tedy vyhrává v těchto případech:
\begin{itemize}
    \item Může vynutit situaci, kdy se hra dostane do pozice, kde je Hráč II na tahu a nemá žádný možný tah (terminální pozice pro Hráče II).
    \item Hra začíná v pozici přiřazené Hráči II, která nemá žádné odchozí hrany – Hráč II okamžitě prohrává a Hráč I vyhrává.
\end{itemize}

\subsection{P-úplnost problému}
\label{sec:GamesCompleteness}

Tento problém je P-úplný \cite{sawa, miyano}. Určení výherní strategie lze provést v polynomiálním čase pomocí tzv. retrográdní analýzy \cite{sawa}, která zpětně vyhodnocuje pozice od koncových uzlů. Algoritmus pracuje iterativně a dokáže zpracovat i grafy s cykly – v případě cyklů bez jednoznačného výsledku přiřadí pozicím status remízy. 

Logika rozhodování funguje následovně: Pro každou pozici určujeme, zda je výherní pro hráče, který je v ní na tahu. Pokud má hráč v dané pozici alespoň jeden tah do pozice, která je výherní pro něj, pak je i aktuální pozice výherní pro něj. Naopak, pokud všechny možné tahy vedou do pozic výherních pro protihráče, pak je aktuální pozice výherní pro protihráče (tedy prohrávající pro hráče na tahu). V případě, že některé tahy vedou do pozic výherních pro protihráče a některé do pozic, jejichž výsledek ještě nebyl určen, zůstává aktuální pozice také nerozhodnuta (remíza).

Implementace využívá frontu (queue) pro postupné zpracování pozic – začínáme koncovými pozicemi a postupně šíříme výsledky směrem k počáteční pozici. Liší se od algoritmu vyhodnocování MCVP (viz kapitola \ref{sec:MCVPEvaluation}) tím, že místo kombinace logických hodnot zde pracujeme s výherními stavy jednotlivých hráčů. I když je problém řešitelný v třídě P, jeho P-úplnost naznačuje, že pravděpodobně neexistuje efektivní paralelní algoritmus – řešení vyžaduje sekvenční zpracování pozic.

\section{Formát vstupu}
\label{sec:GamesInput}

Aplikace nabízí tři způsoby zadávání herního grafu:

\begin{itemize}
    \item \textbf{Interaktivní editace:} Uživatel vytváří a upravuje graf pomocí grafického editoru (viz sekce \ref{sec:GamesInteractive}).
    \item \textbf{Generování náhodných her:} Automatické vytvoření náhodného grafu podle zadaných parametrů (viz sekce \ref{sec:GamesGeneration}).
    \item \textbf{Předpřipravené sady:} Načtení ukázkových připravených příkladů (viz sekce \ref{sec:GamesPreparedSets}).
\end{itemize}

Všechny metody využívají komponentu \texttt{DisplayGraph} pro vizualizaci herního grafu (viz sekce \ref{sec:GamesVisualization}).  

\section{Interaktivní editace grafu}
\label{sec:GamesInteractive}

Komponenta \texttt{ManualInput} umožňuje vytvářet a upravovat herní grafy pomocí interaktivního grafického editoru. Uživatel může:

\begin{itemize}
    \item \textbf{Přidávat uzly:} Vytvořit novou pozici.
    
    \item \textbf{Upravovat uzly:} Kliknutím na uzel ho označíme. Označený uzel můžeme:
    \begin{itemize}
        \item Změnit hráče který je na tahu (Hráč I nebo Hráč II)
        \item Odstranit pozici z grafu
        \item Nastavit jako počáteční pozici
        \item Použít jako zdroj nebo cíl pro vytvoření hrany
        \item Vytvořit hranu z nebo do tohoto uzlu
        \item Odstranit hrany k nebo od tohoto uzlu
    \end{itemize}
            
    \item \textbf{Reorganizovat graf:} Uzly lze přesouvat myší pro lepší vizuální uspořádání.
\end{itemize}

Aplikace průběžně validuje graf. Uživatel je upozorněn, pokud není nastavena počáteční pozice, což je nutné pro spuštění analýzy.

\section{Generování náhodných her}
\label{sec:GamesGeneration}

Modul \texttt{Generator.js} obsahuje funkci \texttt{generateGraph()} pro vytváření náhodných herních grafů. Uživatel nastavuje dva parametry:

\begin{itemize}
    \item \textbf{Počet pozic:} Kolik uzlů (herních pozic) bude graf obsahovat.
    \item \textbf{Pravděpodobnost hrany:} Hodnota 0\% -- 100\% určující, jak pravděpodobné je vytvoření hrany mezi dvěma uzly.
\end{itemize}

Algoritmus generování probíhá ve dvou fázích:

\begin{enumerate}
    \item \textbf{Vytvoření kostry:} Vytvoříme uzly očíslované 0 až \( n-1 \) a každému náhodně přiřadíme hráče. Pro každý uzel \( i > 0 \) pak vytvoříme hranu z náhodného předchozího uzlu (s indexem menším než \( i \)) do uzlu \( i \). Tím zajistíme, že uzel 0 (počáteční pozice) může dosáhnout všechny ostatní uzly, a současně vytvoříme acyklickou kostru grafu.
    
    \item \textbf{Přidání dalších hran:} Pro každou dvojici uzlů \( i, j \) (kde \( i \neq j \)), podle dané pravděpodobnosti přidáme hranu \( i \to j \), pokud ještě neexistuje. Hrany mohou být přidány v libovolném směru, což umožňuje vznik cyklů v grafu.
\end{enumerate}

Výsledný graf je vždy souvislý s uzlem 0 jako počáteční pozicí. Graf může obsahovat cykly, což odpovídá reálným herním situacím, kde se hra může dostat do opakujících se pozic. Algoritmus analýzy (viz sekce \ref{sec:GamesAlgorithm}) je navržen tak, aby správně zpracoval i grafy s cykly a přiřadil těmto pozicím status remízy, pokud nelze jednoznačně určit výherce.

\section{Předpřipravené sady}
\label{sec:GamesPreparedSets}

Ve složce \texttt{Sady/CombinatorialGame} najdeme předpřipravené herní grafy různé velikosti a složitosti.

\subsection{Ukládání a načítání her}
\label{sec:GamesSerialization}

Herní grafy lze exportovat a importovat ve formátu JSON pomocí komponenty \texttt{FileTransferControls}. Formát obsahuje:

\begin{itemize}
    \item Pole \texttt{nodes} s uzly – každý má ID a přiřazeného hráče
    \item Pole \texttt{edges} s hranami ve formátu source-target
    \item \texttt{startingPosition} určující ID počáteční pozice
\end{itemize}

Tento formát umožňuje sdílení her mezi uživateli a vytváření knihovny předpřipravených příkladů.

\section{Vizualizace grafu}
\label{sec:GamesVisualization}

Vizualizace herního grafu využívá komponentu \texttt{DisplayGraph}, která využívá knihovnu \emph{react-force-graph-2d}. Graf zobrazuje:

\begin{itemize}
    \item \textbf{Uzly:} Každý uzel reprezentuje herní pozici. Pod uzlem je zobrazen hráč, který je na tahu (I nebo II). Při najetí myší na libovolný uzel se v centrech všech uzlů zobrazí jejich ID pro snadnější orientaci.
    
    \item \textbf{Počáteční pozice:} Označena oranžovou barvou.
    
    \item \textbf{Hrany:} Možné tahy jsou zobrazeny jako směrované hrany mezi uzly. Hrany patřící do výherní strategie jsou zvýrazněny tlustší čarou a žlutou barvou.
\end{itemize}

Graf používá fyzikální simulaci pro automatické rozmístění uzlů. Uživatel může uzly přesouvat myší a graf přibližovat nebo oddalovat kolečkem myši.

\section{Algoritmus analýzy}
\label{sec:GamesAlgorithm}

Řešení problému kombinatorických her je implementováno v modulu \texttt{ComputeWinner.js}. Algoritmus využívá iterativní přístup (retrográdní analýzu) pro postupné označování pozic od koncových uzlů směrem k počáteční pozici.

\subsection{Vyhodnocení výherních pozic}
\label{sec:GamesRetrograde}

Algoritmus určuje, zda je daná pozice výherní pro hráče, který je v ní na tahu, nebo zda skončí remízou (v případě cyklů bez vynuceného výsledku). Vyhodnocení probíhá následovně:

\begin{enumerate}
    \item \textbf{Koncové pozice:} Pozice bez dalších tahů jsou prohrávající pro hráče, který je v nich na tahu.
    
    \item \textbf{Zpětné šíření:} Od koncových pozic se postupně propagují výsledky:
    \begin{itemize}
        \item Pokud existuje tah do prohrávající pozice soupeře, aktuální pozice je vyhrávající.
        \item Pokud tentýž hráč pokračuje v tahu (bez změny hráče), tah do vyhrávající pozice znamená, že i původní pozice je vyhrávající.
        \item Pokud všechny tahy vedou do vyhrávajících pozic soupeře, aktuální pozice je prohrávající.
    \end{itemize}
    
    \item \textbf{Neurčené pozice:} Pozice, jejichž status nelze jednoznačně určit (například kvůli cyklům), zůstávají označeny jako remíza.
\end{enumerate}

Časová složitost je \( O(V + E) \), kde \( V \) je počet uzlů a \( E \) počet hran.

\subsection{Optimální tahy}
\label{sec:GamesOptimalMoves}

Funkce \texttt{getOptimalMoves()} identifikuje hrany, které jsou součástí výherní strategie. Hrana z pozice \( u \) do pozice \( v \) je optimální, pokud obě pozice jsou výherní pro Hráče I. Tyto hrany jsou zvýrazněny ve vizualizaci a ukazují uživateli cestu k výhře.

\section{Krokové vyhodnocení}
\label{sec:GamesStepByStep}

Pro detailnější průchod grafem implementuje komponenta \texttt{StepByStepGame} krokovatelnou analýzu. Algoritmus zaznamenává každý krok aktualizace stavu grafu:

\begin{itemize}
    \item \textbf{Inicializace:} Všechny pozice jsou na začátku ve stavu REMÍZA.
    \item \textbf{Terminální stavy:} Identifikace pozic bez tahů (PROHRA).
    \item \textbf{Aktualizace:} Postupné šíření výherních a prohrávajících stavů grafem.
    \item \textbf{Vysvětlení:} U každého kroku je zobrazeno vysvětlení, proč došlo ke změně stavu (např. „Z pozice X lze táhnout do prohrávající pozice Y").
\end{itemize}

Uživatel může procházet kroky analýzy pomocí navigačních tlačítek. Aktuálně aktualizovaný uzel je v grafu zvýrazněn. Tato funkce pomáhá pochopit, jak algoritmus postupně řeší hru a jak se vypořádává s cykly.

\chapter{Neprázdnost bezkontextových gramatik}
\label{sec:Grammars}

\section{Teoretický základ}
\label{sec:GrammarsTheory}

Problém neprázdnosti jazyka bezkontextové gramatiky (CFG Non-emptiness Problem) je dalším příkladem P-úplného problému \cite{sawa}. Bezkontextová gramatika představuje formální systém sloužící k generování jazyka, který hraje klíčovou roli v definování syntaxe programovacích jazyků i v analýze neprogramovacích jazyků.

Formálně je gramatika definována jako čtveřice \( G = (N, \Sigma, P, S) \) \cite{uti_grammars}, kde:

\begin{itemize}
    \item \( N \) je konečná množina neterminálních symbolů (neterminálů).
    \item \( \Sigma \) je konečná množina terminálních symbolů (terminálů), disjunktní s \( N \).
    \item \( P \) je konečná množina přepisovacích pravidel tvaru \( A \to \alpha \), kde \( A \in N \) a \( \alpha \in (N \cup \Sigma)^* \).
    \item \( S \in N \) je počáteční symbol (start symbol).
\end{itemize}

Problém spočívá v tom, zda daná gramatika generuje alespoň jedno slovo složené pouze z terminálních symbolů:

\begin{itemize}
    \item \textbf{Vstup:} Bezkontextová gramatika \( G = (N, \Sigma, P, S) \).
    \item \textbf{Výstup:} Rozhodnutí, jestli je jazyk \( L(G) = \{ w \in \Sigma^* \mid S \Rightarrow^* w \} \) neprázdný.
\end{itemize}

Symbol \( \Rightarrow^* \) označuje derivaci v nula nebo více krocích \cite{uti_grammars}. Gramatika generuje neprázdný jazyk právě tehdy, když z počátečního symbolu \( S \) lze odvodit alespoň jedno slovo složené pouze z terminálních symbolů.

\subsection{P-úplnost problému}
\label{sec:GrammarsCompleteness}

Problém neprázdnosti jazyka bezkontextové gramatiky je P-úplný \cite{sawa, miyano}. Tento problém lze vyřešit v polynomiálním čase pomocí algoritmu iterativního označování produktivních neterminálů. Produktivní neterminál je takový, ze kterého lze derivovat řetězec složený pouze z terminálů \cite{sawa}.

I přes řešitelnost v polynomiálním čase je problém P-úplný, což naznačuje obtížnou paralelizovatelnost. Produktivita jednoho neterminálu často závisí na produktivitě jiných neterminálů. Tato vzájemná závislost vyžaduje sekvenční zpracování podobné vyhodnocování MCVP obvodu.

\section{Formát vstupu}
\label{sec:GrammarsInput}

Aplikace nabízí tři způsoby zadávání bezkontextové gramatiky:

\begin{itemize}
    \item \textbf{Manuální zadání:} Uživatel definuje gramatiku pomocí textového pole, kde se zadá celá gramatika najednou (viz sekce \ref{sec:GrammarsManual}).
    \item \textbf{Generování náhodných gramatik:} Automatické vytvoření náhodné gramatiky podle zadaných parametrů (viz sekce \ref{sec:GrammarsGeneration}).
    \item \textbf{Předpřipravené sady:} Načtení ukázkových příkladů gramatik (viz sekce \ref{sec:GrammarsPreparedSets}).
\end{itemize}

Všechny formy vstupu využívají stejnou komponentu pro zobrazení derivačního stromu gramatiky implementovanou ve třídě \texttt{Grammar} (modul \texttt{Grammar.js}).

\section{Manuální zadání gramatiky}
\label{sec:GrammarsManual}

Komponenta \texttt{ManualInput} poskytuje rozhraní ve formě textového okna pro přímé zadávání pravidel gramatiky.

\begin{itemize}
    \item \textbf{Syntaktický formát:} Každé pravidlo se zapisuje na nový řádek ve tvaru \texttt{LS -> PS}. Pro oddělení více alternativ na pravé straně se používá symbol \texttt{|}. Symboly v rámci jedné alternativy musí být odděleny mezerami.
    
    \item \textbf{Kategorizace symbolů:} Algoritmus v modulu \texttt{GrammarParser.js} automaticky rozpoznává typy symbolů. Symboly složené čistě z velkých písmen (včetně českých znaků s diakritikou) jsou považovány za neterminály. Jakékoliv jiné řetězce jsou identifikovány jako terminály.
    
    \item \textbf{Epsilon pravidla:} Pro vyjádření prázdného řetězce ($\epsilon$) lze použít přímo slovo \texttt{epsilon} nebo řecké písmeno.
    
    \item \textbf{Počáteční symbol:} První neterminál uvedený na levé straně prvního pravidla je automaticky nastaven jako počáteční symbol \( S \).
\end{itemize}

Příklad vstupu:
\begin{verbatim}
S -> a S b | epsilon
\end{verbatim}

Aplikace během zpracování textu provádí validaci a kontroluje správnost formátu pravidel. Validace upozorňuje na tyto typy chyb:

\begin{itemize}
    \item \textbf{Chybějící levá nebo pravá strana:} Pravidlo musí obsahovat jak levou stranu (neterminál), tak pravou stranu. Parser zobrazí chybu, pokud v pravidle chybí šipka \texttt{->} nebo pokud je některá strana prázdná.
    
    \item \textbf{Neplatný neterminál:} Levá strana pravidla musí být složena pouze z velkých písmen (včetně českých znaků s diakritikou).
    
    \item \textbf{Zapomenutá mezera:} Parser detekuje pravděpodobné chyby, jako je zapomenutá mezera mezi neterminálem a terminálem (např. \texttt{aS}), což by bylo chybně identifikováno jako jeden terminální symbol. V takovém případě parser navrhne správné oddělení symbolů mezerami.
\end{itemize}

\section{Generování náhodných gramatik}
\label{sec:GrammarsGeneration}

Modul \texttt{GrammarGenerator.js} obsahuje funkci \texttt{generateGrammar()} pro vytváření náhodných bezkontextových gramatik na základě vstupních parametrů. Komponenta \texttt{GenerateInput} poskytuje uživatelské rozhraní s možností nastavení základních i pokročilých parametrů.

\subsection{Základní parametry}
\label{sec:GrammarsGenerationBasic}

Základní parametry umožňují rychlé vygenerování gramatiky a určují její velikost a strukturu. Prvním parametrem je počet neterminálů v rozmezí 1 až 10. Druhým parametrem je počet terminálů, taktéž v rozmezí 1 až 10. Třetí parametr určuje maximální délku pravé strany, zde se mohou zadat hodnoty v rozmezí 1 až 5, přičemž délka každého pravidla je náhodně zvolena v intervalu od 1 do hodnoty tohoto parametru.

\subsection{Pokročilé parametry}
\label{sec:GrammarsGenerationAdvanced}

Pokročilé parametry umožňují větší kontrolu nad vlastnostmi generované gramatiky:

\begin{itemize}
    \item \textbf{Počet pravých stran na neterminál:} 
    \begin{itemize}
        \item Min (1-10): Minimální počet pravidel pro každý neterminál
        \item Max: Maximální počet pravidel, automaticky omezen na rozmezí od hodnoty parametru Min do 10
        \item Pro každý neterminál se náhodně zvolí počet pravidel v tomto rozmezí
    \end{itemize}
    
    \item \textbf{Rekurze:} Kontrola nad rekurzivními pravidly:
    \begin{itemize}
        \item Levá rekurze: Povoluje pravidla tvaru \( A \to A \alpha \)
        \item Pravá rekurze: Povoluje pravidla tvaru \( A \to \alpha A \)
        \item Pokud jsou obě vypnuté, generátor vytváří pouze pravidla bez rekurze
        \item Pravděpodobnost aplikace rekurze je 30\% při povoleném režimu
    \end{itemize}
    
    \item \textbf{Generování epsilon (\( \epsilon \)):} Tři režimy:
    \begin{itemize}
        \item \textbf{Ne:} Negenerují se žádná epsilon pravidla
        \item \textbf{Náhodně:} Epsilon pravidla se generují s pravděpodobností 8\%
        \item \textbf{Vždy:} Epsilon pravidla se generují s pravděpodobností 15\%, přičemž je garantován alespoň jeden výskyt v gramatice
    \end{itemize}
\end{itemize}

\subsection{Algoritmus generování}
\label{sec:GrammarsGenerationAlgorithm}

Algoritmus generování probíhá v těchto krocích:

\begin{enumerate}
    \item \textbf{Vytvoření symbolů:} Generují se pole neterminálů a terminálů podle zadaných počtů. První neterminál je automaticky nastaven jako počáteční symbol \( S \) gramatiky.
    
    \item \textbf{Generování pravidel pro každý neterminál:} 
    \begin{enumerate}
        \item Náhodně se určí počet pravidel pro tento neterminál (v rozmezí vstupních parametrů - min-max)
        \item Pro každé pravidlo:
        \begin{itemize}
            \item Kontrola epsilon režimu – jestli je povolená generace epsilon terminálu, je zde šance na jeho vytvoření (podle zvoleného režimu)
            \item Jinak určení náhodné délky pravé strany (1 až vstupní parametr max délka pravé strany)
            \item Náhodný výběr symbolů – terminály s pravděpodobností 50\%, neterminály s 50\%
            \item Aplikace rekurze – přidání neterminálu na začátek/konec dle nastavení
        \end{itemize}
    \end{enumerate}
    
    \item \textbf{Sestavení gramatiky:} Vytvoření instance třídy \texttt{Grammar} s vygenerovanými symboly a pravidly.
\end{enumerate}

\section{Předpřipravené sady}
\label{sec:GrammarsPreparedSets}

Ve složce \texttt{Sady/Grammar} jsou uloženy předpřipravené příklady gramatik. Tyto sady slouží ke kontrolované demonstraci specifických příkladů.

\subsection{Ukládání a načítání gramatik}
\label{sec:GrammarsSerialization}

Gramatiky lze exportovat a importovat ve formátu JSON pomocí komponenty \texttt{FileTransferControls}. Formát obsahuje:

\begin{itemize}
    \item Pole \texttt{terminals} se seznamem terminálních symbolů
    \item Pole \texttt{nonTerminals} se seznamem neterminálních symbolů
    \item Pole \texttt{productions} s pravidly, kde každé pravidlo má \texttt{ls} (levou stranu) a \texttt{ps} (pravou stranu jako pole symbolů)
    \item \texttt{startSymbol} určující počáteční neterminál
\end{itemize}

Tento formát umožňuje sdílení gramatik mezi uživateli a vytváření předpřipravených příkladů.



\section{Algoritmus vyhodnocení}
\label{sec:GrammarsAlgorithm}

Modul \texttt{GrammarEvaluator.js} obsahuje implementaci algoritmu pro vyhodnocení neprázdnosti gramatiky. Algoritmus zjišťuje, zda je počáteční symbol \( S \) produktivní – tedy zda z něj lze odvodit slovo složené pouze z terminálů nebo prázdný řetězec \( \epsilon \).

\subsection{Identifikace produktivních neterminálů}
\label{sec:GrammarsProductive}

Algoritmus pracuje iterativně a postupně označuje produktivní neterminály \cite{sawa}:

\begin{enumerate}
    \item \textbf{Inicializace:} Vytvoříme prázdnou množinu produktivních symbolů \( \mathcal{P} \).
    
    \item \textbf{Základní krok:} Do \( \mathcal{P} \) přidáme všechny neterminály \( A \), které mají pravidlo \( A \to w \), kde \( w \in \Sigma^* \) (pravá strana obsahuje pouze terminály nebo je prázdná).
    
    \item \textbf{Iterativní rozšiřování:} Opakovaně procházíme všechna pravidla. Pokud existuje pravidlo \( A \to X_1 X_2 \dots X_k \), kde všechny symboly \( X_i \) jsou buď terminály nebo již jsou v množině \( \mathcal{P} \), přidáme \( A \) do \( \mathcal{P} \).
    
    \item \textbf{Ukončení:} Opakujeme krok 3, dokud se množina \( \mathcal{P} \) mění.
    
    \item \textbf{Výsledek:} Pokud \( S \in \mathcal{P} \), pak jazyk \( L(G) \) není prázdný. V opačném případě je \( L(G) = \emptyset \).
\end{enumerate}

Implementace používá optimalizovanou verzi s frontou (work-list algorithm), což odpovídá principu algoritmu průchodu do šířky (BFS). Místo procházení všech pravidel v každém cyklu udržujeme frontu nově označených produktivních neterminálů a kontrolujeme pouze ta pravidla, která tyto neterminály obsahují na pravé straně.

Časová složitost algoritmu je \( O(|P| \cdot l) \) \cite{uti_grammars}, kde \( |P| \) značí celkový počet pravidel a \( l \) je maximální délka pravé strany pravidla. Každé pravidlo je totiž kontrolováno nejvýše jednou pro každý symbol na pravé straně.

\subsection{Rekonstrukce derivace}
\label{sec:GrammarsWitness}

%clean

Algoritmus během svého běhu zaznamenává pro každý produktivní neterminál všechna jeho produktivní pravidla. Na rozdíl od základní verze algoritmu, která ukládá pouze první nalezené produktivní pravidlo, implementace uchovává všechna produktivní pravidla pro každý neterminál. Tato data jsou ukládána do mapy \texttt{allWitnesses}, která pro každý produktivní neterminál obsahuje pole všech jeho produktivních pravých stran.

Po dokončení analýzy produktivních neterminálů je možné provést sestavení derivačního stromu, pokud je počáteční symbol \( S \) produktivní. Sestavení probíhá rekurzivně:

\begin{enumerate}
    \item \textbf{Výběr pravidla:} Pro každý neterminál náhodně vybereme jedno z jeho produktivních pravidel. Při větší hloubce derivace (> 20 kroků) preferujeme nerekurzivní pravidla, aby se předešlo nekonečným derivacím.
    
    \item \textbf{Limitace hloubky:} Rekurzivní konstrukce stromu je omezena na maximální hloubku 30 úrovní. Pokud je tato hloubka překročena, vrátí se chyba a místo vykreslení stromu se zobrazí informační zpráva.
    
    \item \textbf{Stavba uzlů:} Strom se buduje rekurzivně funkcí \texttt{buildNode()}:
    \begin{itemize}
        \item Pro každý neterminál vytvoříme uzel s unikátním identifikátorem
        \item Náhodně vybereme jedno z jeho produktivních pravidel
        \item Pro každý symbol na pravé straně pravidla rekurzivně vytvoříme potomky (zvýšíme hloubku o 1)
        \item Terminály a \( \epsilon \) vytvoří listy stromu bez dalších potomků
    \end{itemize}
    
    \item \textbf{Extrakce slova:} Funkce \texttt{extractTerminals()} prochází strom zleva doprava a čte terminální listy, čímž získá řetězec vygenerovaný danou derivací.
\end{enumerate}

Tato rekonstrukce slouží nejen jako důkaz neprázdnosti jazyka, ale také jako vizuální pomůcka pro pochopení struktury generovaných slov. Náhodný výběr pravidel znamená, že při každém vyhodnocení může být vygenerováno jiné slovo.

\section{Vizualizace derivačního stromu}
\label{sec:GrammarsVisualization}

Pokud gramatika generuje neprázdný jazyk, aplikace nejen potvrdí tento fakt, ale také vizuálně ukáže důkaz vykreslením derivačního stromu pro jedno z možných slov.

\subsection{Komponenta pro vykreslení}
\label{sec:GrammarsTreeComponent}

Pro vykreslení derivačního stromu je využit algoritmus z komponenty \texttt{DerivationTreeVisual}, která využívá knihovnu \emph{react-force-graph-2d}. Strom je zobrazen ve stromovém rozložení pomocí režimu \texttt{dagMode="td"} (top-down).

\section{Krokové vyhodnocení}
\label{sec:GrammarsStepByStep}

Pro detailnější pochopení algoritmu implementuje aplikace krokovatelnou analýzu prostřednictvím komponenty \texttt{StepByStepGrammar}. Tato komponenta využívá modul \texttt{GrammarStepEvaluator.js} k simulaci algoritmu a zobrazuje průběh výpočtu v modálním okně.

\subsection{Struktura zobrazení}
\label{sec:GrammarsStepByStepStructure}

Rozhraní krokového vyhodnocení je rozčleněno do několika sekcí:

\begin{itemize}
    \item \textbf{Seznam pravidel gramatiky:} Zobrazuje všechna produktivní pravidla. Aktuálně kontrolované pravidlo je vizuálně zvýrazněno žlutým pozadím a oranžovým ohraničením. Pravidla, jejichž levá strana je již produktivní, jsou označena zeleným štítkem s textem „Produktivní".
    
    \item \textbf{Množina produktivních symbolů:} Aktuální obsah množiny \( \mathcal{P} \) je zobrazen formou zelených štítků s písmenem neterminálu. Tato množina se postupně rozšiřuje během jednotlivých iterací algoritmu, za podmínky že gramatika obsahuje produktivní neterminály.
    
    \item \textbf{Vysvětlení kroku:} Textový popis aktuální operace algoritmu, například „Pravidlo \( A \to a \) obsahuje pouze terminály. \( A \) je produktivní" nebo „Pravidlo se stalo produktivním díky \( B \). \( A \) přidán mezi produktivní".
\end{itemize}

\subsection{Navigace a ovládání}
\label{sec:GrammarsStepByStepNavigation}

Pro pohyb mezi jednotlivými kroky analýzy jsou k dispozici čtyři navigační tlačítka:

\begin{itemize}
    \item \textbf{Začátek:} Přesun na první krok algoritmu (inicializace prázdné množiny \( \mathcal{P} \))
    \item \textbf{Předchozí:} Krok zpět v historii výpočtu
    \item \textbf{Další:} Posun na následující krok
    \item \textbf{Konec:} Skok na finální krok s výsledkem analýzy
\end{itemize}

Aktuální pozice je indikována čítačem ve formátu „Krok \( x \) z \( n \)", kde \( n \) představuje celkový počet kroků. Tato interaktivní vizualizace umožňuje sledovat, jak algoritmus postupně identifikuje produktivní neterminály na základě již známých produktivních symbolů a jak závislosti mezi pravidly určují pořadí jejich vyhodnocení.
\chapter{Závěr}
\label{sec:Conclusion}

Cílem této práce bylo vytvořit interaktivní webovou aplikaci pro vizualizaci a demonstraci převoditelnosti a řešení P-úplných problémů na příkladu tří vybraných P-úplných problémů: Monotone Circuit Value Problem, kombinatorických her na grafu a problému neprázdnosti jazyka bezkontextové gramatiky.

\section{Dosažené výsledky}
\label{sec:ConclusionAchievements}

Výsledná aplikace splňuje všechny stanovené cíle a poskytuje jednotné prostředí pro práci se třemi P-úplnými problémy. Pro každý problém je implementováno kompletní řešení zahrnující:

\begin{itemize}
    \item \textbf{Flexibilní vstupní systém:} Uživatelé mohou zadávat instance problémů třemi způsoby – manuálním zadáním, generováním náhodných instancí nebo pomocí předpřipravených příkladů. Tyto možnosti umožňují jak experimentování s vlastními příklady, tak rychlé testování algoritmu na náhodných datech.
    
    \item \textbf{Vizualizaci řešení:} Každý problém je doprovázen grafickou reprezentací, která využívá knihovnu \emph{react-force-graph-2d} pro interaktivní zobrazení grafových struktur. Uživatel může manipulovat se zobrazenými grafy, přibližovat je a přesouvat uzly pro lepší orientaci.
    
    \item \textbf{Krokové vyhodnocení:} Implementace krokovatelného průchodu algoritmů umožňuje sledovat každý jednotlivý krok výpočtu s textovým vysvětlením.
    
    \item \textbf{Export a import dat:} Možnost ukládání a načítání instancí problémů ve formátu JSON podporuje sdílení příkladů a vytváření knihoven testovacích případů.
\end{itemize}

Aplikace taktéž demonstruje dva konkrétní převody z MCVP na kombinatorickou hru a na bezkontextovou gramatiku. Oba převody jsou implementovány krokovatelně, takže uživatel může sledovat, jak se jednotlivé uzly obvodu transformují na odpovídající struktury v cílovém problému. Tato vizualizace názorně ilustruje techniku polynomiálních redukcí a vzájemnou převoditelnost P-úplných problémů.

\section{Vzdělávací přínos}
\label{sec:ConclusionEducational}

Aplikace představuje vzdělávací nástroj pro výuku teorie složitosti s interaktivním přístupem. Krokové vyhodnocení s textovými vysvětleními umožňuje pochopit fungování algoritmů na konkrétních příkladech, zatímco vizualizace převodů demonstruje vzájemnou převoditelnost P-úplných problémů. Generování náhodných instancí pak podporuje experimentální učení a pozorování chování algoritmů na různých strukturách vstupů.

\section{Možnosti dalšího rozvoje}
\label{sec:ConclusionFutureWork}

Přestože aplikace poskytuje funkční implementaci všech plánovaných funkcí, existuje prostor pro další rozšíření:

\begin{itemize}
    \item \textbf{Další P-úplné problémy:} Aplikace by mohla být rozšířena o další P-úplné problémy, jako je například vyhodnocování booleovských formulí v konjunktivní normální formě nebo problém dosažitelnosti v grafech s omezenou šířkou.
    
    \item \textbf{Více převodů:} Implementace dalších směrů převodů – například z kombinatorických her na gramatiky nebo opačným směrem z gramatik na MCVP – by poskytla kompletnější obraz vzájemné převoditelnosti těchto problémů.
    
    \item \textbf{Výkonnostní optimalizace:} Pro velmi velké instance (stovky uzlů) by mohly být implementovány optimalizace vykreslování a výpočtu, například pomocí virtualizace zobrazení nebo progresivního vykreslování.
\end{itemize}

Výsledná aplikace může sloužit jako doplněk k tradičním výukovým materiálům v kurzech teorie složitosti a teoretické informatiky, kde pomáhá studentům lépe pochopit abstraktní koncepty prostřednictvím konkrétních interaktivních příkladů.

\endinput


% Seznam literatury
\printbibliography[title={Literatura}, heading=bibintoc]

% Prilohy
% \appendix

% Priloha vlozena primo do hlavniho LaTeX souboru. Ne vsechny prilohy je nutne mit ve zvlastnich souborech.
% \chapter{Dlouhý zdrojový kód}
% \lstinputlisting[label=src:CppExternal,caption={Dlouhý zdrojový kód v jazyce C++ načtený s externího souboru}]{SourceCodes/ArraySortingAlgorithms.cpp}

\end{document}
